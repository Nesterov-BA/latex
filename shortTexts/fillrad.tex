\documentclass[11pt,twoside
]{article}
\usepackage{myrussian}

\addbibresource{refs.bib}
\begin{document}

\section{Радиус заполнения}
\begin{theorem}[Старшие \( n \)-мерные гомологии]
  \label{thm:homology}
  Let $M$ be a connected (not necessarily compact) $n$--dimensional
  manifold without boundary.
  The $n$-dimensional homology $H_n(M;\Z/2)$ of $M$ is
  well-understood and is dictated by the compactness of $M$.
  More specifically,
  \begin{equation}\label{eq:fund-cls}
    H_n(M;\Z/2)\simeq
    \begin{cases}
      \Z/2&\text{ if }M\text{ is compact} \\
      0&\text{ if }M\text{ is not compact}.
    \end{cases}
  \end{equation}
\end{theorem}
Рассмотрим компактное риманово многообразие $M$, изометрически
погруженное в пространство функций $L^{\infty}(M)$ вложением
Куратовского. Будем обозначать это пространство $M ^{\prime}$.
Известно, что $H_{n}(M,\mathbb{Z}_{2})= H_{n}(M ^{\prime},
\mathbb{Z}_{2})= \mathbb{Z}_2$ Далее, будем рассматривать
рассматривать открытые раздутия $B(M ^{\prime}, r)$ пространства $M
^{\prime}$. Для малых $r$ такое раздутие будет гомотопически
эквивалентно пространству $M ^{\prime}$ и $H_{n}\big(B(M
^{\prime},r), \mathbb{Z}_{2}\big) = H_{n}(M ^{\prime},
\mathbb{Z}_{2})= \mathbb{Z}_2$. При этом, раздутие, очевидно будет
открытым, значит не компактным. Но его размерность не будет равна \(
n \), значит
для него теорема \ref{thm:homology} не применима. При этом, если
радиус будет достаточно большим, раздутие станет стягиваемым, и его
группа гомологий станет нулевой. Инфинум таких радиусов, что группа
\( n \)-мерных гомологий раздутия \( H_{n}\big(B(M
^{\prime},r), \mathbb{Z}_{2}\big) \) равна \( 0 \) называется
\emph{радиусом заполнения} \( M \) и обозначается \( \FillRad(M)\). В
\cite{katz1983filling} доказано,
что оценкой сверху является \( \frac 1 3 \diam M \).

В оригинальном определении рассматривается фундаментальный класс, и
его образ при вложении в раздутие. \textbf{Фундаментальный класс} ---
генератор группы \( n \) - мерных гомологий, т.е. класс циклов,
факторизованный по границам. При этом, при вложении в раздутие он
может как остаться нетривиальным, так и перейти в \( 0 \). Это и
означает, что группа гомологий может стать нулевой.
\section{Стягиваемость при большом радиусе}
Будем доказывать, что пространство \( M ^\prime \) стягиваемо в
пространстве \( B(M
^{\prime},r) \). Это и означает, что фундаментальный класс становится
нулевым. В \cite{katz1983filling} приводится доказательство
посредством построения конуса над \( M^\prime \), лежащего целиком в
раздутии. Далее будет изложен упрощенный вариант этой теоремы с более
грубой оценкой.
\begin{theorem}
  Для всякого компактного многообразия \( M \) выполняется \(
  \FillRad(M) \leq \diam M \).
\end{theorem}
\begin{proof}
  Рассмотрим открытый шар радиуса \( \diam M + \varepsilon \) с центром в
  точке \( m \in M^\prime\). По определению диаметра, все
  пространство \( M^\prime \) будет лежать в этом шаре. Шар стягиваем
  в себе, значит любое его подмножество также будет стягиваемо в этом
  шаре. Значит \( M^\prime \) стягиваемо в шаре \( B(m,\diam
  M^\prime) \), и значит и в \( B(M^\prime, \diam M^\prime + \varepsilon) \).
\end{proof}
\section{Гомотопическая эквивалентность при малых радиусах}
Поскольку вложение \( M \rightarrow M^\prime \) является изометрией,
шары \( B(m_1,\varepsilon), B(m_2, \varepsilon )\) одновременно
пересекаются и не пересекаются и в \( M \) и в \( M^\prime \). Для \(
M \) известно, что его комплекс Чеха радиуса меньше радиуса
выпуклости гомотопически эквивалентен \( M \).
Тогда у раздутия, если рассматривать его как семейство шаров, будет
тот же комплекс Чеха, т.е. оно тоже будет ему гомотопически
эквивалентно, а значит и эквивалентно пространству \( M^\prime\).
\printbibliography
\end{document}
