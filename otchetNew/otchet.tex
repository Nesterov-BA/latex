\documentclass{article}
\usepackage{graphicx} % Required for inserting images
\graphicspath{ {./images/} }
\usepackage{amsmath,amsthm,amssymb}
\usepackage{mathtext}
\usepackage[T1,T2A]{fontenc}
\usepackage[utf8]{inputenc}
\usepackage[english,russian]{babel}
\usepackage{multirow}

\DeclareMathOperator{\facmin}{facmin}
\DeclareMathOperator{\facmax}{facmax}
\DeclareMathOperator{\fac}{fac}
\DeclareMathOperator{\tol}{tol}
\DeclareMathOperator{\err}{err}
\title{отчет 29}
\author{nesterov.boris123 }
\date{December 2023}

\begin{document}

\section{Условие.}
\begin{gather*}
 \int\limits_{0}^{\frac{\pi} 2}\big(\dot{x}^{2} - x^{2}\cos(\alpha x)\big)dx \rightarrow ext\\
  x(0) = 0\\
  x(\frac{\pi} 2) = 1\\
  \alpha \in \{0;0.01;0.5;1.5;10.5\}
\end{gather*}
\section{Система необходимых условий оптимальности}
\begin{gather*}
  L = \lambda_{0}\big(\dot{x}^{2} - x^{2}\cos(\alpha x)\big)
\end{gather*}
Уравнение Эйлера-Лагранжа:
\begin{gather*}
  \ddot{x} = \frac{\alpha}{2} x^{2}\sin\alpha x - x \cos \alpha x
\end{gather*}
\section{Краевая задача.}
Итак, получили краевую задачу:
$$\begin{cases}
  \dot x_{1} = x_{2}\\
  \dot{x_{2}} = \frac{\alpha}{2} x_{1}^{2}\sin\alpha x_{1} - x_{1} \cos \alpha x_{1}\\
  x_1(0) = 0\\
  x(\frac{\pi} 2) = 1
\end{cases}$$
\section{Аналитическое решение при $\lambda = 0$}
При $\alpha = 0$ получаем $\ddot{x} = -x$, откуда из граничных условий:
\[x = \sin t\]
Отсюда в частности получаем, что точное решение при $\alpha = 0$ начинается в $(0,1)$.
\section{Численное решение в общем виде}
Зададим функцию ошибок как функцию от $x_{2}(0)$, возвращающую $x_{1}(\frac{\pi}{2}-1)$ --- ошибка.
Для $\alpha = 0$ известно, что она равна нулю в $1$. Для малых $\alpha$ достаточно искать нули
 методом Ньютона, начиная из 1. Для $\alpha = 10.5$ метод Ньютона не
 сходится. Воспользуемся методом случайного поиска, и получим, например, что один из нулей функции ошибок лежит на
 интервале $(1.7, 1.9)$. Метод Ньютона из точки $1.8$ сходится. Также из значений функции ошибок на интервале $(0, 5)$
 ясно, что нулей несколько, т. е. допустимых экстремалей несколько при $\alpha = 10.5$.


\section{Оценки точности решений.}
Матрица Якоби системы:
$$A = \begin{pmatrix}
0 & 1  \\
  2\alpha x_{1} \sin \alpha x_{1} + (\frac\alpha 2 x^{2}_{1}-1)\cos \alpha x_{1}  & 0
\end{pmatrix}.$$
Для нахождения ее логарифмической нормы необходимо найти максимальное по модулю собственное значение матрицы:
$$\frac 1 2 (A + A^{T}) = \begin{pmatrix}
0 & \alpha x_{1} \sin \alpha x_{1} + (\frac\alpha 4 x^{2}_{1}-\frac 1 2)\cos \alpha x_{1} + \frac 1 2 \\
  \alpha x_{1} \sin \alpha x_{1} + (\frac\alpha 4 x^{2}_{1}-\frac 1 2)\cos \alpha x_{1} + \frac 1 2  & 0
\end{pmatrix}.$$
Им будет $\lambda(x) = |\alpha x_{1} \sin \alpha x_{1} + (\frac\alpha 4 x^{2}_{1}-\frac 1 2)\cos \alpha x_{1}|$.
Ошибка расчитывалась по формуле $\delta(x_{i+1}) = r_{i} + \delta(x_{i})*e^{L_{i}}$, где
\begin{itemize}
  \item $\delta(x_{N})$ -- глобольная ошибка
  \item $r_{i}$ -- ошибка на шаге, равная $10^{-11}$
  \item $L_{i} = h\max(\lambda(x_{i}), \lambda(x_{i-1}))$
\end{itemize}
\section{Исследование оптимальности экстремалей}
Отметим, что $L_{\dot{x}\dot{x}} = 2 > 0$, значит выполняется усиленное условие Лежандра
\begin{gather*}
L_{\dot{x}x} = 0 \\
L_{xx} =2\alpha x_{1} \sin \alpha x_{1} + (\frac\alpha 2 x^{2}_{1}-1)\cos \alpha x_{1}
\end{gather*}
\[\int\limits_{0}^{\frac \pi 2} \dot{h}^{2} + \alpha \hat{x} (\sin \alpha \hat{x} + (\frac\alpha 4 \hat{x}^{2}-\frac 1 2)\cos \alpha \hat x)h^{2}\]
Уравнение Якоди имеет вид:
        \[\ddot{h} = \alpha \hat{x} (\sin \alpha \hat{x} + (\frac\alpha 4 \hat{x}^{2}-\frac 1 2)\cos \alpha \hat x)h\]
        \[h(0) = 0.\]

Условие Якоби выполнено, если не существует не равного $0$ $h$, такого что $h(\tau) = 0$, $\tau \in (0, \frac \pi 2)$.
Для усиленного условия таких $\tau$ должно не быть в $(0, \frac \pi 2]$.

При $\alpha = 0$, уравнение Якоби приобретает вид
\begin{gather*}
  \ddot h = -h\\
  h(0) = 0
\end{gather*}
Решением этого уравнения явяется $C\sin t$, у которого нет сопряженных точек на $(0, \frac \pi 2]$, т. е. выполнено усилонное
условие Якоби.

Функция Вейерштрасса будет иметь вид $(u + v)^{2}\ge 0$, т. е. выполнено усиленное условие Вейерштрасса.

Итак, при $\alpha = 0$ экстремаль является сильным локальным минимумом. Так как задача квадратичная, минимум будет глобальным.
\section{Результаты вычислений.}
В таблице приведены $\alpha$, начальные параметры и векторы ошибок:

\begin{tabular}{|c|c|c|c|}
  \hline
   $\alpha$ & $x_{2}(0)$ & $x_{1}(1) - 1$ & Ошибка\\
  \hline
    $0$ & $1$ & $2e-10$ & $ 5e-11$\\
    $0.01$ & $0.999963$ &$4e-11$ &$ 5e-11$ \\
    $0.5$ &$0.923263$ &$9e-10$ & $3e-11$ \\
    $1.5$ &$0.682993$ &$3e-09$& $6e-11$\\
    $10.5$ &$1.797965$ &$-4e-09$& $2e-6$\\
  \hline
\end{tabular}
\end{document}
