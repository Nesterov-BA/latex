

\documentclass[11pt,twoside,draft
]{article}
\usepackage{amsmath,amsfonts,amssymb,amsthm,indentfirst,enumerate,textcomp}
\usepackage[T1,T2A]{fontenc}
\usepackage[utf8]{inputenc}
\usepackage{chebsb}
\usepackage[russian]{babel}
\usepackage{indentfirst, array}
\usepackage{amscd,latexsym}
\usepackage{mathrsfs}
%\usepackage[ruled, linesnumbered]{algorithm2e}
\usepackage{tabularx}
\usepackage{multirow}

\usepackage{graphicx}
\usepackage{textcase}% Оформление страниц


\label{beg}
% заглавие стати и аннотация
\levkolonttl{%левый колонтитул - авторы
Б.~А.~Нестеров}
\prvkolonttl{%правый колонтитул - сокращенное название статьи
О расстоянии Громова–Хаусдорфа \ldots}

\UDK{% УДК статьи
514}

\DOI{
10.22405/2226-8383-\god-\tom-\iss-\pageref{beg}-\pageref{end}
}

\title
{%название статьи на русском языке с указанием источника финансирования при необходимости
О расстоянии Громова--Хаусдорфа между облаком ограниченных метрических пространств и облаком с нетривиальной стационарной группой 
}
{%название статьи на английском языке
On the Gromov--Hausdorff distance between the cloud of bounded
metric spaces and a cloud with nontrivial stabilizer}

\author
{%авторы статьи
Б. А. Нестеров}
{%авторы статьи
B. A. Nesterov}

\Position{
  профессор
}
\AcademicDegree{
  д.ф.-м.н.
}
\SciAdvisor{
  Тужилин Алексей Августинович
}
\Cite{
Б. А. Нестеров. О расстоянии Громова--Хаусдорфа между облаком ограниченных метрических пространств и облаком с нетривиальной стационарной группой // Чебышевcкий сборник, \god, т.~\tom, вып.~\iss, с.~\pageref{beg}--\pageref{end}.
}
{Nesterov B. A. \god, ``On the Gromov--Hausdorff distance between the cloud of bounded
metric spaces and cloud with nontrivial stabilizer''\,, {\it Che\-by\-shev\-skii sbornik}, vol.~\tom, no.~\iss, pp.~\pageref{beg}--\pageref{end}.
}

\info
{%авторы статьи на русском языке
\noindent {\bf Нестеров Борис Аркадьевич}~\\
\noindent Кафедра дифференциальной геометрии и приложений\\
\noindent {\bf Научный руководитель:}\\
\noindent профессор, д.ф.-м.н.\\
\noindent Тужилин Алексей Августинович
}
{%авторы статьи на английском языке
\noindent {\bf Nesterov Boris Arkadyevich}~--- Lomonosov Moscow State University (Moscow).

\noindent
\emph{e-mail: nesterov.boris123@gmail.com}

}

\Abstract
{%Аннотация статьи на русском языке 150-250 слов с учетом ключевых слов
В статье обсуждается класс всех метрических пространств, рассматриваемых с точностью до 
нулевого расстояния Громова--Хаусдорфа между ними. Этот класс разбивается на облака --- классы пространств, лежащих на конечном расстоянии от данного. В работе доказывается, что каждое облако является собственным классом. Между облаками естественно определяется расстояние Громова--Хаусдорфа, по аналогии с метрическими пространствами. В работе показано, что при некоторых ограничениях расстояние между облаком ограниченных метрических пространств и облаком с нетривиальной стационарной группой равно бесконечности. В частности, посчитано расстояние между облаком ограниченных метрических пространств и облаком содержащим вещественную прямую.
}
{%Аннотация статьи на английском языке
The paper studies the class of all metric spaces considered up to zero Gromov-Hausdorff distance between them. In this class, we examine clouds --- classes of spaces situated at finite Gromov-Hausdorff distances from a reference space. The paper proves that all clouds are proper classes. The Gromov--Hausdorff distance is defined for clouds analogous to the case of metric spaces. The paper shows that under certain limitations the distance between the cloud of bounded metric spaces and a cloud with a nontrivial stabilizer is finite. In particular, the distance between the cloud of bounded metric spaces and the cloud containing the real line is calculated.
}

\keywords
{%ключевые слова на русском языке
метрические пространства, расстояние Громова--Хаусдорфа, облака, собственный класс
}
{%ключевые слова на английском языке
metric spaces, Gromov–Hausdorff distance, clouds, proper class
}

%число наименований в библиографии
\Bibliography{15 названий.}{15 titles.}
\DeclareMathOperator{\supp}{supp}
\DeclareMathOperator{\dis}{dis}
\DeclareMathOperator{\St}{St}
\DeclareMathOperator{\diam}{diam}
\DeclareMathOperator{\thin}{thin}
\begin{document}


%генерация заглавия статьи
\maketitle
\newpage

\section{Введение}  \addcontentsline{toc}{section}{Введение}
Расстояние Громова--Хаусдорфа, впервые введенное в \cite{Edwards} позволяет
рассматривать класс метрических пространств, как псевдометрический и изучать
свойства его геометрии. Традиционно, изучаются свойства компактных
метрических пространств, для которых расстояние Громова--Хаусдорфа становится
метрикой \cite{gromov_structures_1981}.

Позднее М. Громов рассматривал в \cite{gromov_metric_2001} класс всех метрических
пространств, не обязательно ограниченных. Одним из свойств этого класса
оказалось то, что он разбивается на классы пространств, лежащих на конечном
расстоянии Громова--Хаусдорфа друг от друга. Эти классы были впоследствии
названы \textbf{облаками}. Одним из вопросов была стягиваемость облаков, по
аналогии со стягиваемостью метрического класса компактных метрических
пространств (\cite{gromov_structures_1981}) было предположено, что все облака
стягиваемы. Для доказательства или опровержения этого факта необходимо было
введение дополнительных понятий.

Понятие непрерывного отображения, необходимое для определения стягиваемости,
требует топологию образа и прообраза, но так как это отображение действует на
собственном классе метрических пространств, возникают затруднения.
Собственный класс по своему определению не может быть элементом другого
класса, следовательно топологию в привычном смысле ввести нельзя. В \cite{borzov_extendability_2020}
 авторы использую понятие фильтрации
множествами для определения аналога топологии на собственном классе.

Далее встает вопрос о непосредственно стягиваемости, то есть о непрерывности
отображения умножения метрического пространства на произвольное положительное
вещественное число. В общем случае, это отображение может быть разрывным,
например при умножении пространства \( \mathbb{Z} \)
(\cite{mikhailov2025newgeodesiclinesgromovhausdorff}), или даже делать
расстояние Громова--Хаусдорфа между образом и прообразом бесконечным
(\cite{TuzhBog1}). Соответственно, если рассматривать операцию умножения на
число, как операцию над облаками, то облако может перейти как в себя, так и в
другое облако. В связи с данным свойством, в \cite{TuzhBog2} было введено
понятие стационарной группу --- мультипликативной группы отображений
умножения на число, переводящих облако само в себя. В этой же работе было
введено понятие центра облака --- пространства переходящего в себя под
действием отображений стационарной группы.

В данной работе рассматривается вопрос расстояния Громова--Хаусдорфа между облаками. В предыдущих работах были доказаны некоторые его свойства, в частности, если стационарные группы двух облаков пересекаются нетривиальным образом, то расстояние Громова--Хаусдорфа между ними или \( 0 \) или бесконечно. В настоящей работе введено понятие угла облака, и показано, что расстояние Громова--Хаусдорфа между двумя облаками с разными углами равно бесконечности при нетривиальном пересечении их стационарных групп.

Автор выражает благодарность своему научному руководителю, Тужилину А.А. и профессору Иванову А.О. за постановку задачи и плодотворное обсуждение результатов.





\section{Углы}  \addcontentsline{toc}{section}{Введение}
Далее считаем, что все облака имеют нетривиальную стационарную группу. \\
В евклидовом пространстве косинус угла между двумя векторами определяется по теореме косинусов:
$\|a-b\|^2 = \|a\|^2 + \|b\|^2 - 2\cos(a,b)\|a\|\|b\|$. Если длины векторов равны, то формула превращается в $\|a-b\|^2=2\|a\|^2 - 2\cos(a,b)\|a\|^2$. Таким образом косинус между векторами будет равен $\frac{\|a-b\|^2}{\|a\|^2}-1$. Этой формулой мотивировано следующее определение:

\begin{defin}
  Пусть у облака $[M]$ центр $M$. Также, пусть метрические пространства $X_1,X_2$ находятся на расстоянии $|X_1,M|=|X_2,M|=r$ и $|X_1,X_2|=d$. \emph{Углом} между метрическими пространствами $X_1,X_2$ называется величина $$\varphi(X_1,X_2) = \arccos\left(1-\frac{d^2}{2r^2}\right).$$
  По неравенству треугольника $0\le d^2 \le 4r^2$,откуда $-1 \le 1-\displaystyle\frac{d^2}{2r^2}\le 1$. Это означает, что угол определен корректно.\\
  \emph{Углом раствора} облака $[M]$ называется величина
  \[
    \varphi([M])=\sup\big\{\varphi(X_1,X_2)\colon |X_1,M|=|X_2,M|\big\}.
  \]
\end{defin}
Приведем известные примеры углов раствора:
\begin{enumerate}
    \item $\varphi\big([\Delta_1]\big)=\frac{\pi}{3}$.
    \item $\varphi\big([\mathbb{R}]\big) = \pi$.
\end{enumerate}
Понятие угла раствора обобщает идею неравенства треугольника и ультраметрического неравенства. Следующая лемма является тривиальным свойством угла.
\begin{lemma}
  Для любого облака $[M]$ и $\lambda \in \mathbb{R}^+$ выполнено
  $\varphi\big(\lambda[M]\big) = \varphi\big([M]\big).$
\end{lemma}
\begin{proof}
  Пусть в облаке $[M]$ есть пространства $X_1,X_2$ и угол между ними равен $\varphi(X_1,X_2)$. Это означает, что $1-\frac{|X_1,X_2|^2}{2|X_1,M|^2}=\cos\big(\varphi(X_1,X_2)\big)$. В облаке $\lambda[M]$ соответственно лежат пространства $\lambda X_1, \lambda X_2$, найдем косинус угла между ними:
  \[
    \cos\big(\varphi(\lambda X_1, \lambda X_2)\big) = 1-\frac{|\lambda X_1,\lambda X_2|^2}{2|\lambda X_1,M|^2} = 1- \frac{\lambda^2|X_1,X_2|^2}{2\lambda^2|X_1,M|^2}=\cos \big(\varphi(X_1,X_2)\big).
  \]
  Отображение $\pi_\lambda \colon X \rightarrow \lambda X$ является биекцией между облаками $[M],\lambda [M]$ и сохраняет углы между пространствами. Значит, углы раствора этих облаков равны.
\end{proof}

\begin{theorem}
  Пусть облака $[X],[Y]$ имеют нетривиальное пересечение стационарных групп и $\varphi\big([X]\big)\neq \varphi\big([Y]\big)$. Тогда $\big|[X],[Y]\big|=\infty$.
\end{theorem}

\begin{proof}
  Пусть для пространств $ X_1,X_2 $ выполнено $ |X,X_1|=|X,X_2| = r $. Тогда эти пространства связаны соотношением $ |X_1,X_2|\le k_Xr   $ 
  Обозначим $ \ $
\end{proof}

  
\end{document}

%библиография по ГОСТу
\begin{thebibliography}{99}

	\bibitem{Edwards} Edwards D. \emph{The Structure of Superspace. In: Studies in Topology}, ed. by Stavrakas N.M. and Allen K.R.//1975, New York, London, San Francisco, Academic Press, Inc.
	\bibitem{Gromov81} Gromov M. \emph{Structures m\'etriques pour les vari\'et\'es riemanniennes}, edited by Lafontaine and Pierre Pansu// 1981.
	\bibitem{Gromov99} Gromov M. \emph{Metric structures for Riemannian and non-Riemannian spaces}// Birkh\"auser, 1999. ISBN 0-8176-3898-9 (translation with additional content).


	\bibitem{memoli2}
	Mémoli F., Gromov-Hausdorff distances in Euclidean spaces // 2008 IEEE Computer Society Conference on Computer Vision and Pattern Recognition Workshops, Anchorage, AK, USA, 2008, pp. 1-8
	\bibitem{TuzhBog1}
	Bogatyy S.A., Tuzhilin A.A. Gromov–Hausdorff class: its completeness and cloud geometry //2021, ArXiv e-prints,
	arXiv:2110.06101, [math.MG]

	\bibitem{Neumann}
	von Neumann J., "Eine Axiomatisierung der Mengenlehre"//1925, Journal für die Reine und Angewandte Mathematik

	\bibitem{Bernays}
	Bernays P., "A System of Axiomatic Set Theory—Part I"//1937, The Journal of Symbolic Logic, doi:10.2307/2268862, JSTOR 2268862

	\bibitem{Godel}
	Gödel K. The Consistency of the Axiom of Choice and of the Generalized Continuum Hypothesis with the Axioms of Set Theory (Revised ed.)//1940, Princeton University Press,  ISBN 978-0-691-07927-1

	\bibitem{BorIvTuzh1}
	Borzov S.I., Ivanov A.O., Tuzhilin A.A., Extendability of Metric Segments in
	Gromov–Hausdorff Distance // 2020, ArXiv e-prints,
	arXiv:2009.00458, [math.MG]

	\bibitem{TuzhBog2}
	Bogataya S.I., Bogatyy S.A., Redkozubov V.V. Tuzhilin A.A. Clouds in Gromov–Hausdorff Class: their
	completeness and centers//2022, ArXiv e-prints,
	arXiv:2202.07337, [math.MG]



	\bibitem{Lectures}
	Бураго Д., Бураго Ю., Иванов С.А. Курс метрической геометрии//2004, Ин-т компьютерных исслед.
	\bibitem{BogBog1}
	Bogataya S.I., Bogatyy S.A. Isometric Cloud Stabilizer//2023, Topology and its Applications,
	Volume 329
	\bibitem{levySet}Levy A. Basic set theory. Perspectives in mathematical logic//1979, Springer-Verlag, Berlin, Heidelberg, and New York





\end{thebibliography}


%библиография по Гарвардскому стандарту
\label{end}

\end{document}
