

\documentclass[11pt,twoside,draft
]{article}
\usepackage{amsmath,amsfonts,amssymb,amsthm,indentfirst,enumerate,textcomp}
\usepackage[T1,T2A]{fontenc}
\usepackage[utf8]{inputenc}
\usepackage{chebsb}
\usepackage[russian]{babel}
\usepackage{indentfirst, array}
\usepackage{amscd,latexsym}
\usepackage{mathrsfs}
%\usepackage[ruled, linesnumbered]{algorithm2e}
\usepackage{tabularx}
\usepackage{multirow}

\usepackage{graphicx}
\usepackage{textcase}% Оформление страниц


\label{beg}
% заглавие стати и аннотация
\levkolonttl{%левый колонтитул - авторы
Б.~А.~Нестеров}
\prvkolonttl{%правый колонтитул - сокращенное название статьи
О расстоянии Громова–Хаусдорфа \ldots}

\UDK{% УДК статьи
514}

\DOI{
10.22405/2226-8383-\god-\tom-\iss-\pageref{beg}-\pageref{end}
}

\title
{%название статьи на русском языке с указанием источника финансирования при необходимости
О расстоянии Громова--Хаусдорфа между облаком ограниченных метрических пространств и облаком с нетривиальной стационарной группой 
}
{%название статьи на английском языке
On the Gromov--Hausdorff distance between the cloud of bounded
metric spaces and a cloud with nontrivial stabilizer}

\author
{%авторы статьи
Б. А. Нестеров}
{%авторы статьи
B. A. Nesterov}

\Position{
  профессор
}
\AcademicDegree{
  д.ф.-м.н.
}
\SciAdvisor{
  Тужилин Алексей Августинович
}
\Cite{
Б. А. Нестеров. О расстоянии Громова--Хаусдорфа между облаком ограниченных метрических пространств и облаком с нетривиальной стационарной группой // Чебышевcкий сборник, \god, т.~\tom, вып.~\iss, с.~\pageref{beg}--\pageref{end}.
}
{Nesterov B. A. \god, ``On the Gromov--Hausdorff distance between the cloud of bounded
metric spaces and cloud with nontrivial stabilizer''\,, {\it Che\-by\-shev\-skii sbornik}, vol.~\tom, no.~\iss, pp.~\pageref{beg}--\pageref{end}.
}

\info
{%авторы статьи на русском языке
\noindent {\bf Нестеров Борис Аркадьевич}~\\
\noindent Кафедра дифференциальной геометрии и приложений\\
\noindent {\bf Научный руководитель:}\\
\noindent профессор, д.ф.-м.н.\\
\noindent Тужилин Алексей Августинович
}
{%авторы статьи на английском языке
\noindent {\bf Nesterov Boris Arkadyevich}~--- Lomonosov Moscow State University (Moscow).

\noindent
\emph{e-mail: nesterov.boris123@gmail.com}

}

\Abstract
{%Аннотация статьи на русском языке 150-250 слов с учетом ключевых слов
В статье обсуждается класс всех метрических пространств, рассматриваемых с точностью до 
нулевого расстояния Громова--Хаусдорфа между ними. Этот класс разбивается на облака --- классы пространств, лежащих на конечном расстоянии от данного. В работе доказывается, что каждое облако является собственным классом. Между облаками естественно определяется расстояние Громова--Хаусдорфа, по аналогии с метрическими пространствами. В работе показано, что при некоторых ограничениях расстояние между облаком ограниченных метрических пространств и облаком с нетривиальной стационарной группой равно бесконечности. В частности, посчитано расстояние между облаком ограниченных метрических пространств и облаком содержащим вещественную прямую.
}
{%Аннотация статьи на английском языке
The paper studies the class of all metric spaces considered up to zero Gromov-Hausdorff distance between them. In this class, we examine clouds --- classes of spaces situated at finite Gromov-Hausdorff distances from a reference space. The paper proves that all clouds are proper classes. The Gromov--Hausdorff distance is defined for clouds analogous to the case of metric spaces. The paper shows that under certain limitations the distance between the cloud of bounded metric spaces and a cloud with a nontrivial stabilizer is finite. In particular, the distance between the cloud of bounded metric spaces and the cloud containing the real line is calculated.
}

\keywords
{%ключевые слова на русском языке
метрические пространства, расстояние Громова--Хаусдорфа, облака, собственный класс
}
{%ключевые слова на английском языке
metric spaces, Gromov–Hausdorff distance, clouds, proper class
}

%число наименований в библиографии
\Bibliography{15 названий.}{15 titles.}
\DeclareMathOperator{\supp}{supp}
\DeclareMathOperator{\dis}{dis}
\DeclareMathOperator{\St}{St}
\DeclareMathOperator{\diam}{diam}
\DeclareMathOperator{\thin}{thin}
\begin{document}


%генерация заглавия статьи
\maketitle
\newpage

\section{Введение}  \addcontentsline{toc}{section}{Введение}
Привет!


\section{Углы}  \addcontentsline{toc}{section}{Введение}
В данной секции мы рассмотрим два определения угла облака. Для того, чтобы дать эти определения сначала необходимо определить угол между пространствами. 
\\
Далее, будем считать, что у облаков нетривиальная стационарная группа и будем обозначать $ [M] $ - облако с центром $ m $.
\begin{defin}
    Пусть у облака $ [M] $ нетривиальная стационарная группа и его центром является $ M $. \textbf{Углом} между пространствами $ X_1, X_2 $, $ |X_1 M| = r_1, | X_2, M | =r_2, |X_1 X_2| = d$, где $ r_1, r_2 \neq 0 $ называется величина $ \arccos \left(\frac{r_1^2 + r_{2}^2 - d^2}{2r_1r_2} \right)$.\\ Будем обозначать его $ \varphi(X_1, X_2) $.
\end{defin}
\begin{remark}
    Такое определение естественным образом вытекает из теоремы косинусов, а именно $ c^{2} = a^2 + b^2 - 2ab\cos (a,b) $.
\end{remark}
У угла между пространствами есть следующее свойство.
\begin{lemma}
    Для любых \( X_1, X_2 \in [M], \lambda \in \St([M]) \) выполняется \( \varphi (X_1, X_2) = \varphi (\lambda X_1, X_2) \).
\end{lemma} 
\begin{proof}
    Как в определении угла, будем обозначать \[ |X_1 M| = r_1, | X_2, M | =r_2, |X_1 X_2| = d.\] 
\end{proof}
Рассмотрим теперь два интересующих нас определения угла облака.
\begin{defin}
\textbf{Углом} облака $ [M] $ называется величина
\[
    \varphi \big([M]\big)= \sup \big\{\varphi (X_1, X_2) \mid | X_1,M |, |X_2,M| \neq 0\big\}.
\]
\end{defin}
\begin{defin}
    \textbf{Равнобедренным углом} облака $ [M] $ называется величина \[ \varphi_e \big([M]\big) = \sup \big\{\varphi (X_1, X_2) \mid | X_1,M | = |X_2,M| \neq 0\big\}. \]
\end{defin}
Приведем известные примеры углов облаков.
\begin{lemma}
    Для облаков \( [\Delta _{1}], [\mathbb{R}] \) известно следующее:
    \begin{itemize}
    \item \( \varphi \big([\Delta _{1}]\big) = \frac{\pi }{2} \),
    \item \( \varphi_e \big([\Delta _{1}]\big) = \frac{\pi }{3} \),
    \item \( \varphi \big([\mathbb{R}]\big) = \varphi_e \big([\mathbb{R}]\big) = \pi  \).
    \end{itemize}
\end{lemma}

%библиография по ГОСТу
\begin{thebibliography}{99}

	\bibitem{Edwards} Edwards D. \emph{The Structure of Superspace. In: Studies in Topology}, ed. by Stavrakas N.M. and Allen K.R.//1975, New York, London, San Francisco, Academic Press, Inc.
	\bibitem{Gromov81} Gromov M. \emph{Structures m\'etriques pour les vari\'et\'es riemanniennes}, edited by Lafontaine and Pierre Pansu// 1981.
	\bibitem{Gromov99} Gromov M. \emph{Metric structures for Riemannian and non-Riemannian spaces}// Birkh\"auser, 1999. ISBN 0-8176-3898-9 (translation with additional content).


	\bibitem{memoli2}
	Mémoli F., Gromov-Hausdorff distances in Euclidean spaces // 2008 IEEE Computer Society Conference on Computer Vision and Pattern Recognition Workshops, Anchorage, AK, USA, 2008, pp. 1-8
	\bibitem{TuzhBog1}
	Bogatyy S.A., Tuzhilin A.A. Gromov–Hausdorff class: its completeness and cloud geometry //2021, ArXiv e-prints,
	arXiv:2110.06101, [math.MG]

	\bibitem{Neumann}
	von Neumann J., "Eine Axiomatisierung der Mengenlehre"//1925, Journal für die Reine und Angewandte Mathematik

	\bibitem{Bernays}
	Bernays P., "A System of Axiomatic Set Theory—Part I"//1937, The Journal of Symbolic Logic, doi:10.2307/2268862, JSTOR 2268862

	\bibitem{Godel}
	Gödel K. The Consistency of the Axiom of Choice and of the Generalized Continuum Hypothesis with the Axioms of Set Theory (Revised ed.)//1940, Princeton University Press,  ISBN 978-0-691-07927-1

	\bibitem{BorIvTuzh1}
	Borzov S.I., Ivanov A.O., Tuzhilin A.A., Extendability of Metric Segments in
	Gromov–Hausdorff Distance // 2020, ArXiv e-prints,
	arXiv:2009.00458, [math.MG]

	\bibitem{TuzhBog2}
	Bogataya S.I., Bogatyy S.A., Redkozubov V.V. Tuzhilin A.A. Clouds in Gromov–Hausdorff Class: their
	completeness and centers//2022, ArXiv e-prints,
	arXiv:2202.07337, [math.MG]



	\bibitem{Lectures}
	Бураго Д., Бураго Ю., Иванов С.А. Курс метрической геометрии//2004, Ин-т компьютерных исслед.
	\bibitem{BogBog1}
	Bogataya S.I., Bogatyy S.A. Isometric Cloud Stabilizer//2023, Topology and its Applications,
	Volume 329
	\bibitem{levySet}Levy A. Basic set theory. Perspectives in mathematical logic//1979, Springer-Verlag, Berlin, Heidelberg, and New York





\end{thebibliography}


%библиография по Гарвардскому стандарту
\label{end}

\end{document}
