Расстояние Громова--Хаусдорфа, впервые введенное в \cite{Edwards} позволяет
рассматривать класс метрических пространств с точностью до изометрии как
псевдометрический и изучать свойства его геометрии. Псевдометрический
означает, что расстояние между различными элементами может быть равно \( 0
\). В дальнейшем мы будем считать, что все метрические пространства
рассматриваются с точностью до нулевого расстояния Громова--Хаусдорфа.
Традиционно, изучаются свойства компактных метрических пространств, для
которых расстояние Громова--Хаусдорфа становится метрикой
(\cite{gromov_structures_1981}).

Все компактные метрические пространства образуют множество. При введении на
них расстояния Громова--Хаусдорфа это множество становится метрическим
пространством. Данное пространство называется \textbf{пространством
Громова--Хаусдорфа}. Пространство Громова--Хаусдорфа стягиваемо, так как
операция умножения метрического пространства на \( \lambda \in
\mathbb{R}^+\), как функция от двух аргументов является непрерывной. Это
означает, что она является гомотопией в пространство \( \Delta _1 \).

Позднее М. Громов рассматривал в \cite{gromov_metric_2001} класс всех
метрических пространств, не обязательно ограниченных. Одним из свойств этого
класса оказалось то, что он разбивается на классы пространств, лежащих на
конечном расстоянии Громова--Хаусдорфа друг от друга. Эти классы были
впоследствии названы \textbf{облаками}. Одним из вопросов была стягиваемость
облаков. По аналогии со стягиваемостью метрического класса ограниченных
метрических пространств было предположено (\cite{gromov_structures_1981}),
что все облака стягиваемы. Для доказательства или опровержения этого факта
необходимо было введение дополнительных понятий.

Понятие непрерывного отображения, необходимое для определения стягиваемости,
требует топологию образа и прообраза, но так как это отображение действует на
собственном классе метрических пространств, возникают затруднения.
Собственный класс по своему определению не может быть элементом другого
класса, следовательно топологию в привычном смысле ввести нельзя. В
\cite{borzov_extendability_2020} авторы используют понятие фильтрации
множествами для определения аналога топологии на собственном классе.

При помощи данной конструкции можно доказать стягиваемость облака
ограниченных метрических пространств по аналогии со случаем пространства
Громова--Хаусдорфа. В данном случае под стягиваемостью понимается
непрерывность отображения умножения пространства на положительное
вещественное число.
% вот тут переписать начало, что стягиваемость  -- это умножение на \lambda
% В случае класса ограниченных метрических пространств можно все
% доказать (по работе, которую скинули). По неравенству треугольника.
% Развитая в (3) техника позволяет перенести все на случай
% ограниченных метрических пространств.
% Капитально переписать
В общем случае, это отображение может быть разрывным, например при умножении
пространства \( \mathbb{Z} \)
(\cite{mikhailov2025newgeodesiclinesgromovhausdorff}). Также, при умножении пространства на \( \lambda \), оно может перейти в пространство на бесконечном расстоянии от изначального (\cite{TuzhBog1}).
Соответственно, если рассматривать операцию умножения на число, как операцию
над облаками, то облако может перейти как в себя, так и в другое облако. В
связи с данным свойством, в \cite{TuzhBog2} было введено понятие стационарной
группы --- мультипликативной группы отображений умножения на число,
переводящих облако само в себя. В этой же работе было введено понятие центра
облака --- пространства, переходящего в себя под действием отображений
стационарной группы.
%сказать про расстояние Громова--Хаусдорфа между облаками и про
% конструкцию соответствий

Одним из способов задания расстояния Громова--Хаусдорфа является конструкция соответствий. При этом, для задания соответствия не обязательно, чтобы классы, между которыми оно определяется были множествами. Это позволяет перенести эту конструкцию на облака и определить для них расстояние Громова--Хаусдорфа.
В данной работе изучается расстояние Громова--Хаусдорфа между облаками. В
предыдущих курсовых работах были доказаны некоторые его свойства, в
частности, если стационарные группы двух облаков пересекаются нетривиальным
образом, то расстояние Громова--Хаусдорфа между ними или \( 0 \) или
бесконечно. В настоящей работе введено понятие угла облака и показано, что
расстояние Громова--Хаусдорфа между двумя облаками с разными углами равно
бесконечности при нетривиальном пересечении их стационарных групп.

Автор выражает благодарность своему научному руководителю, Тужилину А.А. и
профессору Иванову А.О. за постановку задачи и плодотворное обсуждение
результатов.
