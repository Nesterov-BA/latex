Расстояние Громова--Хаусдорфа, впервые введенное в \cite{Edwards} позволяет
рассматривать класс метрических пространств, как псевдометрический и изучать
свойства его геометрии. Традиционно, изучаются свойства компактных
метрических пространств, для которых расстояние Громова--Хаусдорфа становится
метрикой \cite{gromov_structures_1981}.

Позднее М. Громов рассматривал в \cite{gromov_metric_2001} класс всех метрических
пространств, не обязательно ограниченных. Одним из свойств этого класса
оказалось то, что он разбивается на классы пространств, лежащих на конечном
расстоянии Громова--Хаусдорфа друг от друга. Эти классы были впоследствии
названы \textbf{облаками}. Одним из вопросов была стягиваемость облаков, по
аналогии со стягиваемостью метрического класса компактных метрических
пространств (\cite{gromov_structures_1981}) было предположено, что все облака
стягиваемы. Для доказательства или опровержения этого факта необходимо было
введение дополнительных понятий.

Понятие непрерывного отображения, необходимое для определения стягиваемости,
требует топологию образа и прообраза, но так как это отображение действует на
собственном классе метрических пространств, возникают затруднения.
Собственный класс по своему определению не может быть элементом другого
класса, следовательно топологию в привычном смысле ввести нельзя. В \cite{borzov_extendability_2020}
 авторы использую понятие фильтрации
множествами для определения аналога топологии на собственном классе.

Далее встает вопрос о непосредственно стягиваемости, то есть о непрерывности
отображения умножения метрического пространства на произвольное положительное
вещественное число. В общем случае, это отображение может быть разрывным,
например при умножении пространства \( \mathbb{Z} \)
(\cite{mikhailov2025newgeodesiclinesgromovhausdorff}), или даже делать
расстояние Громова--Хаусдорфа между образом и прообразом бесконечным
(\cite{TuzhBog1}). Соответственно, если рассматривать операцию умножения на
число, как операцию над облаками, то облако может перейти как в себя, так и в
другое облако. В связи с данным свойством, в \cite{TuzhBog2} было введено
понятие стационарной группу --- мультипликативной группы отображений
умножения на число, переводящих облако само в себя. В этой же работе было
введено понятие центра облака --- пространства переходящего в себя под
действием отображений стационарной группы.

В данной работе рассматривается вопрос расстояния Громова--Хаусдорфа между облаками. В предыдущих работах были доказаны некоторые его свойства, в частности, если стационарные группы двух облаков пересекаются нетривиальным образом, то расстояние Громова--Хаусдорфа между ними или \( 0 \) или бесконечно. В настоящей работе введено понятие угла облака, и показано, что расстояние Громова--Хаусдорфа между двумя облаками с разными углами равно бесконечности при нетривиальном пересечении их стационарных групп.

Автор выражает благодарность своему научному руководителю, Тужилину А.А. и профессору Иванову А.О. за постановку задачи и плодотворное обсуждение результатов.



