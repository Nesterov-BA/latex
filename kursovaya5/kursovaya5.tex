

\documentclass[11pt,twoside
]{article}
\usepackage{amsmath,amsfonts,amssymb,amsthm,indentfirst,enumerate,textcomp}
\usepackage[T1,T2A]{fontenc}
\usepackage[utf8]{inputenc}
\usepackage{chebsb}
\usepackage[russian]{babel}
\usepackage{indentfirst, array}
\usepackage{amscd,latexsym}
\usepackage{mathrsfs}
%\usepackage[ruled, linesnumbered]{algorithm2e}
\usepackage{tabularx}
\usepackage{multirow}

\usepackage{graphicx}
\usepackage{textcase}% Оформление страниц

\label{beg}
% заглавие стати и аннотация
\levkolonttl{%левый колонтитул - авторы
Б.~А.~Нестеров}
\prvkolonttl{%правый колонтитул - сокращенное название статьи
Углы облаков \ldots}

\DeclareMathOperator{\supp}{supp}
\DeclareMathOperator{\dis}{dis}
\DeclareMathOperator{\St}{St}
\DeclareMathOperator{\diam}{diam}
\DeclareMathOperator{\thin}{thin}
\begin{document}

%генерация заглавия статьи
\begin{titlepage}
    \centering
    \textbf{Московский государственный университет\\
    имени М.В.~Ломоносова}\\[1em]
    Механико-математический факультет\\
    Кафедра дифференциальной геометрии и приложений\\[10em]

    \begin{large}
    {\bfseries КУРСОВАЯ РАБОТА}\\[2em]

    Углы облаков и расстояние Громова--Хаусдорфа между облаками с различными углами\\[0.5em]
    Angles of clouds and the Gromov--Hausdorff distance between clouds with different angles
\end{large}

\vfill
    \begin{flushright}
    Выполнил студент 5 курса\\
    Нестеров Б. А.\\[0.5em]
    Научный руководитель \\ 
    д.ф.-м.н., проф. А.А.~Тужилин
    \end{flushright}

    \vfill
    Москва 2024
  \end{titlepage}%
\newpage
\tableofcontents
\newpage

\section{Введение}
Привет!

\section{Предварительные результаты}
Пусть $X$ и $Y$ ---
метрические пространства. Тогда между ними можно задать расстояние, называемое
расстоянием Громова--Хаусдорфа. Введем два его эквивалентных
определения \cite{Lectures}.
\begin{defin}
  Пусть $X$, $Y$ --- метрические пространства. \emph{Соответствием}
  $R$ между этими пространствами
  называется сюръективное многозначное отображение между ними.
  Множество всех соответствий между $X$ и $Y$ обозначается
  $\mathcal{R}(X,Y)$. Также будем отождествлять соответствие и его график.
\end{defin}
\begin{defin}
  Пусть $R$ --- соответствие между $X$ и $Y$.
  \emph{Искажением} соответствия $R$ является величина
  $$ \dis{R} = \sup{\Bigl\{ \big| |xx'| - |yy'| \big| : (x, y), (x',
  y') \in R\Bigr\}}.$$
  Тогда \emph{расстояние Громова--Хаусдорфа} $d_{GH}(X,Y)$ можно определить
  следующим образом
  $$ d_{GH}(X,Y) = \frac{1}{2}\inf \bigl\{\dis{R} : R \in
  \mathcal{R}(X,Y)\bigr\}.$$
  \label{defSootvet}
\end{defin}

Далее, расстояние Громова--Хаусдорфа между метрическими
пространствами $X$ и $Y$ будет обозначаться $|XY|$ или \( | X,Y | \).

Рассмотрим собственный класс всех метрических пространств и
отождествим в нем между собой все метрические пространства,
находящиеся на нулевом расстоянии друг от друга. Обозначим
получившийся класс $\mathcal{GH}_0$.  На нем расстояние
Громова--Хаусдорфа будет являться обобщенной метрикой. Здесь
''обобщенная'', означает, что расстояние между метрическими
пространствами может быть равно бесконечности.
\begin{defin}[\cite{TuzhBog1}] В классе $\mathcal{GH}_{0}$ рассмотрим следующее
  отношение: $X \thicksim Y \Leftrightarrow d_{GH}(X, Y) < \infty$. Нетрудно
  убедиться, что оно будет отношением эквивалентности. Классы этой
  эквивалентности
  называются \emph{облаками}. Облако, в котором лежит метрическое
  пространство $X$
  будем обозначать $[X]$.
\end{defin}

Для любого метрического пространства $X$ определена операция умножения его
на положительное вещественное число $\lambda\colon X\mapsto \lambda X$, а именно
расстояние между любыми точками
пространства умножается на $\lambda$.
\begin{remark} Пусть метрические
  пространства $X$, $Y$ лежат в одном облаке. Тогда
  $d_{GH}(\lambda X, \lambda Y) = \lambda d_{GH}(X,Y) < \infty$, т.е.
  пространства
  $\lambda X$, $\lambda Y$ также будут лежать в одном облаке.
  \label{remOneCloud}
\end{remark}
\begin{defin}Определим операцию умножения облака $[X]$ на
  положительное вещественное
  число $\lambda$ как отображение, переводящее все пространства $Y \in [X]$ в
  пространства $\lambda Y$. По замечанию $\ref{remOneCloud}$ все
  полученные пространства будут
  лежать в облаке $[\lambda X]$.
\end{defin}
При таком отображении облако может как
измениться, как было показано в \cite{TuzhBog2}, так и перейти в
себя. Для последнего случая вводится специальное
определение.
\begin{defin}[\cite{TuzhBog2}]
  Стационарной группой $\St\bigl([X]\bigr)$ облака $[X]$
  называется подмножество $\mathbb{R}_+$, то есть множества всех
  положительных вещественных чисел, такое, что для всех
  $\lambda \in \St\bigl([X]\bigr)$, $[X] = [\lambda X]$. Полученное подмножество
  действительно будет подгруппой в $\mathbb{R}_+$. Тривиальной
  будем называть стационарную группу, равную $\{1\}$. Пересечение
  двух стационарных групп называем нетривиальным, если оно не равно \( \{1\} \).
\end{defin}

\begin{lemma}[\cite{TuzhBog2}]
  В каждом облаке с нетривиальной стационарной группой существует единственное
  пространство $X$ такое, что для любого $\lambda$ из стационарной группы
  выполняется $X = \lambda X$.
  \label{centerLemma}
\end{lemma}
\begin{defin}
  Пространство из леммы \ref{centerLemma} будем называть \emph{центром} облака.
\end{defin}
\begin{remark} В
  облаке $[\Delta_1]$ для любого пространства $X$ выполняется:
  $$|\lambda X \mu X| = |\lambda - \mu||X\Delta_1|.$$
\end{remark}
\begin{remark}[Ультраметрическое неравенство] В облаке
  $[\Delta_{1}]$ для всех пространств $X_{1}, X_{2}$ выполняется неравенство:
  $$|X_{1}X_{2}| \le \max\big\{|X_{1} \Delta_{1}|,|X_{2}\Delta_{1}|\big\}.$$
  \label{remUltraMetric}
\end{remark}


\section{Два определения угла}
В данной секции мы рассмотрим два определения угла облака. Для того, чтобы дать эти определения сначала необходимо определить угол между пространствами. 
\\
Далее, будем считать, что у облаков нетривиальная стационарная группа и будем обозначать $ [M] $ - облако с центром $ m $.
\begin{defin}
    Пусть у облака $ [M] $ нетривиальная стационарная группа и его центром является $ M $. \textbf{Углом} между пространствами $ X_1, X_2 $, $ |X_1 M| = r_1, | X_2, M | =r_2, |X_1 X_2| = d$, где $ r_1, r_2 \neq 0 $ называется величина $ \arccos \left(\frac{r_1^2 + r_{2}^2 - d^2}{2r_1r_2} \right)$.\\ Будем обозначать его $ \varphi(X_1, X_2) $.
\end{defin}
\begin{remark}
    Такое определение естественным образом вытекает из теоремы косинусов, а именно $ c^{2} = a^2 + b^2 - 2ab\cos (a,b) $.
\end{remark}
У угла между пространствами есть следующее свойство.
\begin{lemma}
    Для любых \( X_1, X_2 \in [M], \lambda \in \St([M]) \) выполняется \( \varphi (X_1, X_2) = \varphi (\lambda X_1, X_2) \).
\end{lemma} 
\begin{proof}
    Как в определении угла, будем обозначать \[ |X_1 M| = r_1, | X_2, M | =r_2, |X_1 X_2| = d.\] 
\end{proof}
Рассмотрим теперь два интересующих нас определения угла облака.
\begin{defin}
\textbf{Углом} облака $ [M] $ называется величина
\[
    \varphi \big([M]\big)= \sup \big\{\varphi (X_1, X_2) \mid | X_1,M |, |X_2,M| \neq 0\big\}.
\]
\end{defin}
\begin{defin}
    \textbf{Равнобедренным углом} облака $ [M] $ называется величина \[ \varphi_e \big([M]\big) = \sup \big\{\varphi (X_1, X_2) \mid | X_1,M | = |X_2,M| \neq 0\big\}. \]
\end{defin}
Приведем известные примеры углов облаков.
\begin{lemma}
    Для облаков \( [\Delta _{1}], [\mathbb{R}] \) известно следующее:
    \begin{itemize}
    \item \( \varphi \big([\Delta _{1}]\big) = \frac{\pi }{2} \),
    \item \( \varphi_e \big([\Delta _{1}]\big) = \frac{\pi }{3} \),
    \item \( \varphi \big([\mathbb{R}]\big) = \varphi_e \big([\mathbb{R}]\big) = \pi  \).
    \end{itemize}
\end{lemma}

\section{Расстояние между облаками с разными углами}
В работе была доказана следующая теорема.
\begin{theorem}
  Пусть в облаке \( [Z] \) есть пространства $Y_{1}$, $
  Y_{2}$ такие, что \\$\max\big\{ |Y_{1}Z|, |Y_{2} Z| \big\} = r>0$, а
  $|Y_{1} Y_{2}|>r$. Тогда, расстояние между облаками $[\Delta_1]$ и $[Z]$
  равно бесконечности. \label{thrmDist}
\end{theorem}
Следующая теорема является обобщение теоремы \ref{thrmDist}.
\begin{theorem}
  Пусть облака \( [M], [N] \) имеют нетривиальное пересечение
  стационарных групп и их углы \( \varphi \big([M]\big), \varphi
  \big([N]\big) \) различны. Тогда \( d_{GH} \big([M], [N]\big) = \infty \).
\end{theorem}
\begin{proof}
  Без ограничения общности будем считать, что \( \varphi
  \big([M]\big)> \varphi \big([N]\big) \). Для доказательства теоремы
  покажем, что не существует соответствия \( R\in
  \mathcal{R}\big([M],[N]\big) \) с конечным искажением.

  Предположим противное, т. е. существует соответствие \( R \) с
  конечным искажением \( \dis R = \varepsilon
  < \infty \) как отображение из облака \( [M] \) в облако \( [N] \).
  Построим функцию \( f \colon [M] \rightarrow [N] \), \( f(X) \) --
  произвольное пространство из \( R(X) \).
  По определению угла, в облаке \( [M] \) найдутся пространства \(
  X_1, X_2 \) для которых выполнено:
  \[
    \varphi \big([N]\big) < \varphi (X_1, X_2) \le \varphi \big([M]\big).
  \]
  Положим \( |X_1 M| = r_1^X,  |X_2 M| = r_2^X, | X_1 X_2 | =
  d^X\). Также зададим функцию
  \[
    d^N \colon \mathbb{R}^+ \times
    \mathbb{R}^+ \rightarrow \mathbb{R}^+, d^N(r_1, r_2) =
    \sqrt{r_1^2 + r_2^2 - 2r_1r_2\cos\Big(\varphi \big([N]\big)\Big)}.
  \]
  Тогда выполняется следующее неравенство:
  \[
    d^N(r_1^X, r_2^X) < d^X.
  \]
  По определению угла для любых пространств \( Y_1, Y_2 \in [N] \) выполняется
  \begin{equation}
    | Y_1 Y_2 | \le d^N\big(| Y_1N |, | Y_2N |\big).
    \label{ineqAngle}
  \end{equation}
  Итак, наша задача --- показать, что найдутся пространства в \( [N] \),
  для которых неравенство (\ref{ineqAngle}) не выполняется.

  Нетривиальность пересечения стационарных групп означает, что
  найдется подгруппа \[
    \{q^k, k\in \mathbb{Z}, q> 1\} \in \St\big([M]\big) \cap \St \big([N]\big).
  \]
  Обозначим \( |N R(M)| = l \). Будем рассматривать пространства \(
  R(q^k X_1), R(q^k X_2), k \in \mathbb{N} \). Для них выполнены
  следующие неравенства:
  \[
    \big| R(q^k X_1), R(q^k X_2) \big| \ge q^k d^X - \varepsilon,
  \]
  \[
    q^k r_1^X - \varepsilon  \le\big | R(M), R(q^k X_1) \big| \le q^k
    r_1^X + \varepsilon ,
  \]
  \[
    q^k r_2^X - \varepsilon  \le\big | R(M), R(q^k X_2) \big| \le q^k
    r_2^X + \varepsilon .
  \]
  Из последних двух неравенств по неравенству треугольника получаем следующее:
  \[
    q^k r_1^X - l - \varepsilon \le \big|N, R(q^k X_1) \big| \le q^k
    r_1^X + l + \varepsilon,
  \]
  \[
    q^k r_2^X - l - \varepsilon \le \big|N, R(q^k X_2) \big| \le q^k
    r_2^X + l + \varepsilon.
  \]
  Поделим неравенства на \( q^k \):
  \[
    \big| q^{-k}R(q^k X_1), q^{-k}R(q^k X_2) \big| \ge d^X -
    \frac{\varepsilon }{q^k},
  \]
  \[
    r_1^X - \frac{l + \varepsilon}{q^k} \le \big|N, q^{-k}R(q^k X_1)
    \big| \le
    r_1^X + \frac{l + \varepsilon}{q^k},
  \]
  \[
    r_2^X - \frac{l + \varepsilon}{q^k} \le \big|N, q^{-k}R(q^k X_2)
    \big| \le
    r_2^X + \frac{l + \varepsilon}{q^k}.
  \]
  Функция \( d^N \) непрерывная, значит
  \[
    \lim_{k \rightarrow \infty
    }d^N\Big(\big|N, q^{-k}R(q^k X_1)
      \big|, \big|N, q^{-k}R(q^k X_2)
    \big|\Big) = d^N(r_1^X, r_2^X).
  \]
  Поэтому существует такое \( k_0 \in \mathbb{N} \), что
  \[
    \big| q^{-k_0}R(q^{k_0} X_1), q^{-k_0}R(q^{k_0} X_2) \big| >
    d^N\Big(\big|N, q^{-k_0}R(q^k_0 X_1)
      \big|, \big|N, q^{-k_0}R(q^{k_0} X_2)
    \big|\Big)
  \]
  Пространства \( q^{-k_0}R(q^{k_0}X_1), q^{-k_0}R(q^{k_0}X_2)\) ---
  искомые, для которых не выполняется неравенство (\ref{ineqAngle}).
  Противоречие.
\end{proof}

\label{end}
\bibliographystyle{acm}
\bibliography{refs}
\end{document}
