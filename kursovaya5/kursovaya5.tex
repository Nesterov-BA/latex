

\documentclass[11pt,twoside
]{article}
\usepackage{amsmath,amsfonts,amssymb,amsthm,indentfirst,enumerate,textcomp}
\usepackage[T1,T2A]{fontenc}
\usepackage[utf8]{inputenc}
\usepackage{chebsb}
\usepackage[russian]{babel}
\usepackage{indentfirst, array, csquotes}
\usepackage{amscd,latexsym}
\usepackage{mathrsfs}
%\usepackage[ruled, linesnumbered]{algorithm2e}
\usepackage{tabularx}
\usepackage{multirow}

\usepackage{graphicx}
\usepackage{textcase}% Оформление страниц
\usepackage[style = ieee]{biblatex}
\addbibresource{refs.bib}

\label{beg}
% заглавие стати и аннотация
\levkolonttl{%левый колонтитул - авторы
Б.~А.~Нестеров}
\prvkolonttl{%правый колонтитул - сокращенное название статьи
Углы облаков \ldots}

\DeclareMathOperator{\supp}{supp}
\DeclareMathOperator{\dis}{dis}
\DeclareMathOperator{\St}{St}
\DeclareMathOperator{\diam}{diam}
\DeclareMathOperator{\thin}{thin}
\begin{document}

%генерация заглавия статьи
\begin{titlepage}
    \centering
    \textbf{Московский государственный университет\\
    имени М.В.~Ломоносова}\\[1em]
    Механико-математический факультет\\
    Кафедра дифференциальной геометрии и приложений\\[10em]

    \begin{large}
    {\bfseries КУРСОВАЯ РАБОТА}\\[2em]

    Углы облаков и расстояние Громова--Хаусдорфа между облаками с различными углами\\[0.5em]
    Angles of clouds and the Gromov--Hausdorff distance between clouds with different angles
\end{large}

\vfill
    \begin{flushright}
    Выполнил студент 5 курса\\
    Нестеров Б. А.\\[0.5em]
    Научный руководитель \\ 
    д.ф.-м.н., проф. А.А.~Тужилин
    \end{flushright}

    \vfill
    Москва 2024
  \end{titlepage}%
\newpage
\tableofcontents
\newpage

\section{Введение}
Расстояние Громова--Хаусдорфа, впервые введенное в \cite{Edwards} позволяет
рассматривать класс метрических пространств, как псевдометрический и изучать
свойства его геометрии. Традиционно, изучаются свойства компактных
метрических пространств, для которых расстояние Громова--Хаусдорфа становится
метрикой \cite{gromov_structures_1981}.

Позднее М. Громов рассматривал в \cite{gromov_metric_2001} класс всех метрических
пространств, не обязательно ограниченных. Одним из свойств этого класса
оказалось то, что он разбивается на классы пространств, лежащих на конечном
расстоянии Громова--Хаусдорфа друг от друга. Эти классы были впоследствии
названы \textbf{облаками}. Одним из вопросов была стягиваемость облаков, по
аналогии со стягиваемостью метрического класса компактных метрических
пространств (\cite{gromov_structures_1981}) было предположено, что все облака
стягиваемы. Для доказательства или опровержения этого факта необходимо было
введение дополнительных понятий.

Понятие непрерывного отображения, необходимое для определения стягиваемости,
требует топологию образа и прообраза, но так как это отображение действует на
собственном классе метрических пространств, возникают затруднения.
Собственный класс по своему определению не может быть элементом другого
класса, следовательно топологию в привычном смысле ввести нельзя. В \cite{borzov_extendability_2020}
 авторы использую понятие фильтрации
множествами для определения аналога топологии на собственном классе.

Далее встает вопрос о непосредственно стягиваемости, то есть о непрерывности
отображения умножения метрического пространства на произвольное положительное
вещественное число. В общем случае, это отображение может быть разрывным,
например при умножении пространства \( \mathbb{Z} \)
(\cite{mikhailov2025newgeodesiclinesgromovhausdorff}), или даже делать
расстояние Громова--Хаусдорфа между образом и прообразом бесконечным
(\cite{TuzhBog1}). Соответственно, если рассматривать операцию умножения на
число, как операцию над облаками, то облако может перейти как в себя, так и в
другое облако. В связи с данным свойством, в \cite{TuzhBog2} было введено
понятие стационарной группу --- мультипликативной группы отображений
умножения на число, переводящих облако само в себя. В этой же работе было
введено понятие центра облака --- пространства переходящего в себя под
действием отображений стационарной группы.

В данной работе рассматривается вопрос расстояния Громова--Хаусдорфа между облаками. В предыдущих работах были доказаны некоторые его свойства, в частности, если стационарные группы двух облаков пересекаются нетривиальным образом, то расстояние Громова--Хаусдорфа между ними или \( 0 \) или бесконечно. В настоящей работе введено понятие угла облака, и показано, что расстояние Громова--Хаусдорфа между двумя облаками с разными углами равно бесконечности при нетривиальном пересечении их стационарных групп.

Автор выражает благодарность своему научному руководителю, Тужилину А.А. и профессору Иванову А.О. за постановку задачи и плодотворное обсуждение результатов.




\section{Предварительные результаты}
Пусть $X$ и $Y$ ---
метрические пространства. Расстояние между точками метрического пространства \( x, x^\prime \) будем обозначать \( | xx^\prime | \). Между метрическими пространствами можно задать расстояние, называемое
расстоянием Громова--Хаусдорфа. Введем два его эквивалентных
определения \cite{Lectures}.
\begin{defin}
  Пусть $X$, $Y$ --- метрические пространства. \emph{Соответствием}
  $R$ между этими пространствами
  называется сюръективное многозначное отображение между ними.
  Множество всех соответствий между $X$ и $Y$ обозначается
  $\mathcal{R}(X,Y)$. Также будем отождествлять соответствие и его график.
\end{defin}
\begin{defin}
  Пусть $R$ --- соответствие между $X$ и $Y$.
  \emph{Искажением} соответствия $R$ является величина
  $$ \dis{R} = \sup{\Bigl\{ \big| |xx'| - |yy'| \big| : (x, y), (x',
  y') \in R\Bigr\}}.$$
  Тогда \emph{расстояние Громова--Хаусдорфа} $d_{GH}(X,Y)$ можно определить
  следующим образом
  $$ d_{GH}(X,Y) = \frac{1}{2}\inf \bigl\{\dis{R} : R \in
  \mathcal{R}(X,Y)\bigr\}.$$
  \label{defSootvet}
\end{defin}

Далее, расстояние Громова--Хаусдорфа между метрическими
пространствами $X$ и $Y$ будет обозначаться $|XY|$ или \( | X,Y | \).

Рассмотрим собственный класс всех метрических пространств и
отождествим в нем между собой все метрические пространства,
находящиеся на нулевом расстоянии друг от друга. Обозначим
получившийся класс $\mathcal{GH}_0$.  На нем расстояние
Громова--Хаусдорфа будет являться обобщенной метрикой. Здесь
<<обобщенная>>, означает, что расстояние между метрическими
пространствами может быть равно бесконечности.
\begin{defin}[\cite{TuzhBog1}] В классе $\mathcal{GH}_{0}$ рассмотрим следующее
  отношение: $X \thicksim Y \Leftrightarrow d_{GH}(X, Y) < \infty$. Нетрудно
  убедиться, что оно будет отношением эквивалентности. Классы этой
  эквивалентности
  называются \emph{облаками}. Облако, в котором лежит метрическое
  пространство $X$
  будем обозначать $[X]$.
\end{defin}

Для любого метрического пространства $X$ определена операция умножения его
на положительное вещественное число $\lambda\colon X\mapsto \lambda X$, а именно
расстояние между любыми точками
пространства умножается на $\lambda$.
\begin{remark} Пусть метрические
  пространства $X$, $Y$ лежат в одном облаке. Тогда
  $d_{GH}(\lambda X, \lambda Y) = \lambda d_{GH}(X,Y) < \infty$, т.е.
  пространства
  $\lambda X$, $\lambda Y$ также будут лежать в одном облаке.
  \label{remOneCloud}
\end{remark}
\begin{defin}Определим операцию умножения облака $[X]$ на
  положительное вещественное
  число $\lambda$ как отображение, переводящее все пространства $Y \in [X]$ в
  пространства $\lambda Y$. По замечанию $\ref{remOneCloud}$ все
  полученные пространства будут
  лежать в облаке $[\lambda X]$.
\end{defin}
При таком отображении облако может как
измениться, как было показано в \cite{TuzhBog2}, так и перейти в
себя. Для последнего случая вводится специальное
определение.
\begin{defin}[\cite{TuzhBog2}]
    \emph{Стационарной группой} $\St\bigl([X]\bigr)$ облака $[X]$
  называется подмножество $\mathbb{R}_+$, то есть множества всех
  положительных вещественных чисел, такое, что для всех
  $\lambda \in \St\bigl([X]\bigr)$, $[X] = [\lambda X]$. Полученное подмножество
  действительно будет подгруппой в $\mathbb{R}_+$. Тривиальной
  будем называть стационарную группу, равную $\{1\}$. Пересечение
  двух стационарных групп называем нетривиальным, если оно не равно \( \{1\} \).
\end{defin}

\begin{lemma}[\cite{TuzhBog2}]
  В каждом облаке с нетривиальной стационарной группой существует единственное
  пространство $X$ такое, что для любого $\lambda$ из стационарной группы
  выполняется $X = \lambda X$.
  \label{centerLemma}
\end{lemma}
\begin{defin}
  Пространство из леммы \ref{centerLemma} будем называть \emph{центром} облака.
\end{defin}
\begin{defin}
    Пространство мощности \( n \), в котором все расстояния между неравными точками равны \( 1 \) называется  \emph{симплексом} мощности \( n \) и обозначается \( \Delta _n \). Так, например, \( \Delta _1 \)~--- одноточечное пространство.
\end{defin}
\begin{remark} В
  облаке $[\Delta_1]$ для любого пространства $X$ выполняется:
  $$|\lambda X \mu X| = |\lambda - \mu||X\Delta_1|.$$
\end{remark}
\begin{remark}[Ультраметрическое неравенство] В облаке
  $[\Delta_{1}]$ для всех пространств $X_{1}, X_{2}$ выполняется неравенство:
  $$|X_{1}X_{2}| \le \max\big\{|X_{1} \Delta_{1}|,|X_{2}\Delta_{1}|\big\}.$$
  \label{remUltraMetric}
\end{remark}


\section{Два определения угла}
Далее считаем, что все облака имеют нетривиальную стационарную группу. \\
В евклидовом пространстве косинус угла между двумя векторами определяется по теореме косинусов:
$\|a-b\|^2 = \|a\|^2 + \|b\|^2 - 2\cos(a,b)\|a\|\|b\|$. Если длины векторов равны, то формула превращается в $\|a-b\|^2=2\|a\|^2 - 2\cos(a,b)\|a\|^2$. Таким образом косинус между векторами будет равен $\frac{\|a-b\|^2}{\|a\|^2}-1$. Этой формулой мотивировано следующее определение:

\begin{defin}
  Пусть у облака $[M]$ центр $M$. Также, пусть метрические пространства $X_1,X_2$ находятся на расстоянии $|X_1,M|=|X_2,M|=r$ и $|X_1,X_2|=d$. \emph{Углом} между метрическими пространствами $X_1,X_2$ называется величина $$\varphi(X_1,X_2) = \arccos\left(1-\frac{d^2}{2r^2}\right).$$
  По неравенству треугольника $0\le d^2 \le 4r^2$,откуда $-1 \le 1-\displaystyle\frac{d^2}{2r^2}\le 1$. Это означает, что угол определен корректно.\\
  \emph{Углом раствора} облака $[M]$ называется величина
  \[
    \varphi([M])=\sup\big\{\varphi(X_1,X_2)\colon |X_1,M|=|X_2,M|\big\}.
  \]
\end{defin}
Приведем известные примеры углов раствора:
\begin{enumerate}
    \item $\varphi\big([\Delta_1]\big)=\frac{\pi}{3}$.
    \item $\varphi\big([\mathbb{R}]\big) = \pi$.
\end{enumerate}
Понятие угла раствора обобщает идею неравенства треугольника и ультраметрического неравенства. Следующая лемма является тривиальным свойством угла.
\begin{lemma}
  Для любого облака $[M]$ и $\lambda \in \mathbb{R}^+$ выполнено
  $\varphi\big(\lambda[M]\big) = \varphi\big([M]\big).$
\end{lemma}
\begin{proof}
  Пусть в облаке $[M]$ есть пространства $X_1,X_2$ и угол между ними равен $\varphi(X_1,X_2)$. Это означает, что $1-\frac{|X_1,X_2|^2}{2|X_1,M|^2}=\cos\big(\varphi(X_1,X_2)\big)$. В облаке $\lambda[M]$ соответственно лежат пространства $\lambda X_1, \lambda X_2$, найдем косинус угла между ними:
  \[
    \cos\big(\varphi(\lambda X_1, \lambda X_2)\big) = 1-\frac{|\lambda X_1,\lambda X_2|^2}{2|\lambda X_1,M|^2} = 1- \frac{\lambda^2|X_1,X_2|^2}{2\lambda^2|X_1,M|^2}=\cos \big(\varphi(X_1,X_2)\big).
  \]
  Отображение $\pi_\lambda \colon X \rightarrow \lambda X$ является биекцией между облаками $[M],\lambda [M]$ и сохраняет углы между пространствами. Значит, углы раствора этих облаков равны.
\end{proof}

\begin{theorem}
  Пусть облака $[X],[Y]$ имеют нетривиальное пересечение стационарных групп и $\varphi\big([X]\big)\neq \varphi\big([Y]\big)$. Тогда $\big|[X],[Y]\big|=\infty$.
\end{theorem}

\begin{proof}
  Пусть для пространств $ X_1,X_2 $ выполнено $ |X,X_1|=|X,X_2| = r $. Тогда эти пространства связаны соотношением $ |X_1,X_2|\le k_Xr   $ 
  Обозначим $ \ $
\end{proof}

  
\end{document}

\section{Расстояние между облаками с разными углами}
В работе \cite{nesterov2025gromovhausdorffdistancecloudbounded} была
доказана следующая теорема.
\begin{thm}
  Пусть в облаке \( [Z] \) есть пространства $Y_{1}$, $ Y_{2}$ такие,
  что \\$\max\big\{ |Y_{1}Z|, |Y_{2} Z| \big\} = r>0$, а $|Y_{1}
  Y_{2}|>r$. Тогда, расстояние между облаками $[\Delta_1]$ и $[Z]$
  равно бесконечности. \label{thrmDist}
\end{thm}
Следующая теорема является обобщением теоремы \ref{thrmDist}.
\begin{thm}
  Пусть облака \( [M], [N] \) имеют нетривиальное пересечение
  стационарных групп и их углы \( \varphi \big([M]\big), \varphi
  \big([N]\big) \) различны. Тогда \( d_{GH} \big([M], [N]\big) =
  \infty \).
\end{thm}
\begin{proof}
  Без ограничения общности будем считать, что \( \varphi \big([M]\big)>
  \varphi \big([N]\big) \). Для доказательства теоремы покажем, что не
  существует соответствия \( R\in \mathcal{R}\big([M],[N]\big) \) с
  конечным искажением.

  Предположим противное, т. е. существует соответствие \( R \) с
  конечным искажением \( \dis R = \varepsilon < \infty \) как
  отображение из облака \( [M] \) в облако \( [N] \). Построим функцию
  \( f \colon [M] \rightarrow [N] \), \( f(X) \) -- произвольное
  пространство из \(
    R(X)
  \). По определению угла, в облаке \( [M] \) найдутся пространства \(
  X_1, X_2 \), для которых выполнено
  \[
    \varphi \big([N]\big) < \varphi (X_1, X_2) \le \varphi
  \big([M]\big). \]
  Положим \( |X_1 M| = r_1^X,  |X_2 M| = r_2^X, | X_1 X_2 | = d^X\).
  Также зададим функцию
  \[ d^N \colon \mathbb{R}^+ \times \mathbb{R}^+ \rightarrow
    \mathbb{R}^+, d^N(r_1, r_2) =
    \sqrt{r_1^2 + r_2^2 - 2r_1r_2\cos\Big(\varphi \big([N]\big)\Big)}.
  \]
  Тогда выполняется следующее неравенство:
  \begin{equation}
    d^N(r_1^X, r_2^X) < d^X.
    \label{ineqdx}
  \end{equation}
  По определению угла для любых пространств \( Y_1, Y_2 \in [N] \)
  выполняется
  \begin{equation}
    | Y_1 Y_2 | \le d^N\big(| Y_1N |, | Y_2N |\big). \label{ineqAngle}
  \end{equation}
  Итак, наша задача --- показать, что найдутся пространства в \( [N]
  \), для которых неравенство (\ref{ineqAngle}) не выполняется.

  Нетривиальность пересечения стационарных групп означает, что найдется
  подгруппа \[
    \{q^k \colon k\in \mathbb{Z}, q> 1\} \in \St\big([M]\big) \cap \St
    \big([N]\big).
  \]
  Положим \( |N R(M)| = l \). Будем рассматривать пространства \(
  R(q^k X_1), R(q^k X_2), k \in \mathbb{N} \). Для них выполнены
  следующие неравенства:
  \begin{equation}
    \big| R(q^k X_1), R(q^k X_2) \big| \ge q^k d^X - \varepsilon,
    \label{ineqDistx1x2}
  \end{equation}
  \[
    q^k r_1^X - \varepsilon  \le\big | R(M), R(q^k X_1) \big| \le q^k
    r_1^X + \varepsilon ,
  \]
  \[
    q^k r_2^X - \varepsilon  \le\big | R(M), R(q^k X_2) \big| \le q^k
    r_2^X + \varepsilon .
  \]
  Из последних двух неравенств по неравенству треугольника получаем
  следующее:
  \begin{equation}
    q^k r_1^X - l - \varepsilon \le \big|N, R(q^k X_1) \big| \le
    q^k r_1^X + l + \varepsilon,\label{ineqDistx1N}
  \end{equation}
  \begin{equation}
    q^k r_2^X - l - \varepsilon \le \big|N, R(q^k X_2) \big| \le q^k
    r_2^X + l + \varepsilon.\label{ineqDistx2N}
  \end{equation}
  Поделим неравенства (\ref{ineqDistx1x2}), (\ref{ineqDistx1N}),
  (\ref{ineqDistx2N}) на \( q^k \):
  \[ \big| q^{-k}R(q^k X_1), q^{-k}R(q^k X_2) \big| \ge d^X -
  \frac{\varepsilon }{q^k}, \]
  \[
    r_1^X - \frac{l + \varepsilon}{q^k} \le \big|N, q^{-k}R(q^k X_1)
    \big| \le
    r_1^X + \frac{l + \varepsilon}{q^k},
  \]
  \[
    r_2^X - \frac{l + \varepsilon}{q^k} \le \big|N, q^{-k}R(q^k X_2)
    \big| \le
    r_2^X + \frac{l + \varepsilon}{q^k}.
  \]
  Функция \( d^N \) непрерывная, значит
  \[
    \lim_{k \rightarrow \infty
    }d^N\Big(\big|N, q^{-k}R(q^k X_1)
      \big|, \big|N, q^{-k}R(q^k X_2)
    \big|\Big) = d^N(r_1^X, r_2^X).
  \]
  Из неравенства (\ref{ineqdx}) следует, что найдутся \( \delta
    _1, \delta _2
  > 0 \) такие, что \( d^N(r_1^X,r_2^X)<d^N(r_1^X, r_2^X)+\delta
  _1 < d^X- \delta _2 < d^X \). Выберем такое \( k_1 \), чтобы
  для всех \( k > k_1 \) выполнялось
  \[
    d^N\Big(\big|N, q^{-k}R(q^k X_1)
      \big|, \big|N, q^{-k}R(q^k X_2)
    \big|\Big) \in \big(d^N(r_1^X, r_2^X)- \delta _1,
  d^N(r_1^X, r_2^X)+ \delta _1\big).  \]
  Также выберем \( k_2 \), такое что \( \big| q^{-k}R(q^k X_1),
  q^{-k}R(q^k X_2) \big| > d^X - \delta _2 \).
  Теперь, если взять \( k_0 = \max\{k_1,k_2\} \), то
  \[ \big| q^{-k_0}R(q^{k_0} X_1), q^{-k_0}R(q^{k_0} X_2) \big|
    > d^N\Big(\big|N, q^{-k_0}R(q^k_0 X_1) \big|, \big|N,
  q^{-k_0}R(q^{k_0} X_2) \big|\Big) \]
  Пространства \( q^{-k_0}R(q^{k_0}X_1), q^{-k_0}R(q^{k_0}X_2)\)
  ---
  искомые, для которых не выполняется неравенство (\ref{ineqAngle}).
  Противоречие.
\end{proof}

\label{end}
\printbibliography
% \bibliographystyle{acm}
% \bibliography{refs}
\end{document}
