В данном разделе мы рассмотрим два определения угла облака. Для того, чтобы
дать эти определения сначала необходимо определить угол между пространствами.
\\
В дальнейшем будем считать, что у облаков нетривиальная стационарная группа и в
обозначении облака $ [M] $, $ M $ является его центром.
\begin{defin}
    \textbf{Углом} между пространствами $ X_1, X_2 \in [M] $ такими, что $ |X_1
    M| = r_1, | X_2 M | =r_2, |X_1 X_2| = d$, где $ r_1,
    r_2 \neq 0 $ называется величина $ \arccos \left(\frac{r_1^2 +
    r_{2}^2 - d^2}{2r_1r_2} \right)$.\\ Будем обозначать его $
    \varphi(X_1, X_2) $.
\end{defin}
\begin{remark}
    Такое определение естественным образом вытекает из евклидовой теоремы
    косинусов, а именно $ c^{2} = a^2 + b^2 - 2ab\cos (a,b) $.
\end{remark}
Прежде, чем формулировать свойство угла, нам понадобится следующая
вспомогательная лемма.
\begin{lemma}
    \label{lem:lambdacloud} Для любого облака \( [M] \) и \( \lambda
    \in\mathbb{R}^+ \) выполняются следующие свойства{\textup{:}}
    \begin{enumerate}
        \item \( \St\big(\lambda [M]\big) = \St\big([M]\big)\),
            \label{lem:lambdacloud:1}
        \item \( \lambda [M] = [\lambda M] \).\label{lem:lambdacloud:2}
    \end{enumerate}
\end{lemma}
\begin{proof}
    \ref{lem:lambdacloud:1}. Пусть \( \mu \in \St\big([M]\big) \). Тогда \( \mu
    \cdot \lambda [M] = \lambda \cdot \mu [M]= \lambda [M] \), т. е. \( \mu \in
    \St\big(\lambda [M]\big) \). Отсюда \( \St\big([M]\big) \subseteq
    \St\big(\lambda [M]\big) \). Обратное включение получается если рассмотреть
    облако \( [M] \) как \(\frac{1}{\lambda}\cdot \lambda [M] \). 

    \ref{lem:lambdacloud:2}. Так как стационарные группы совпадают, для любого
    \( \mu \in  \St\big(\lambda [M])\) выполняется \( \mu \cdot \lambda M =
        \lambda \cdot \mu M = \lambda M \). Значит, \( \lambda M \) - центр
        облака \( \lambda [M] \).
    \end{proof}
    У угла между пространствами есть следующее свойство.
    \begin{lemma}
        \label{lemmaAngleBetweenSpaces} Для любых \( X_1, X_2 \in [M], \lambda
        \in\mathbb{R}^+ \) выполняется \( \varphi (X_1, X_2) = \varphi (\lambda
        X_1,
        \lambda X_2) \).
    \end{lemma}
    \begin{proof}
        По лемме \ref{lem:lambdacloud} \( \lambda M \) будет центром \( \lambda
        [M]
        \). Как в определении угла, положим \[ |X_1 M| = r_1, | X_2, M | =r_2,
            |X_1
        X_2| = d.\] По определению угла получаем следующую цепочку
        равенств: \[ \varphi (\lambda X_1, \lambda X_2) = \frac{\lambda
            ^{2} r_1^2 + \lambda ^2 r_2^2 - \lambda^2 d^2}{2 \lambda  r_1
        \lambda r_2} =
        \varphi (X_1, X_2).
    \]
    На этом доказательство закончено.
\end{proof}
Рассмотрим теперь два интересующих нас определения угла облака.
\begin{defin}
    \textbf{Углом} облака $ [M] $ называется величина
    \[
        \varphi \big([M]\big)= \sup \big\{\varphi (X_1, X_2) \colon | X_1,M
        |, |X_2,M| \neq 0\big\}.
    \]
\end{defin}
\begin{defin}
    \textbf{Равнобедренным углом} облака $ [M] $ называется величина \(
    \varphi_e \big([M]\big) = \sup \big\{\varphi (X_1, X_2) \bigm| |X_1,M |
    = |X_2,M| \neq 0\big\}. \)
\end{defin}
Для этих определений получаем следствие из леммы \ref{lemmaAngleBetweenSpaces}.
\begin{lemma}
    Для любого облака \( [M] \) и \( \lambda \in\mathbb{R}^+ \)
    выполняется\textup{:}\label{lem:angles}
    \begin{enumerate}
        \item \(\varphi \big(\lambda [M]\big) = \varphi
            \big([M]\big)\),\label{lem:angles:1}
        \item \(\varphi_e \big(\lambda [M]\big) = \varphi_e
            \big([M]\big)\),\label{lem:angles:2}
        \item \( \varphi_e \big([M]\big) \le \varphi \big([M]\big)
            \),\label{lem:angles:3}
        \item \(0 \le \varphi ([M]) \le \pi\),\label{lem:angles:4}
        \item \(0 \le \varphi_e ([M]) \le \pi\).\label{lem:angles:5}
    \end{enumerate}
\end{lemma}
\begin{proof}
    \ref{lem:angles:1}. Следствие леммы \ref{lemmaAngleBetweenSpaces}.

    \ref{lem:angles:2}. Следствие леммы \ref{lemmaAngleBetweenSpaces}.

    \ref{lem:angles:3}. В определении равнобедренного угла супремум берется по
    подмножеству пространств из определения угла, следовательно равнобедренный
    угол не больше.

    \ref{lem:angles:4}.  Следует из определения арккосинуса и неравенства
    треугольника.

    \ref{lem:angles:5}. Аналогично.
\end{proof}
Приведем примеры углов облаков.
\begin{lemma}
    Для облаков \( [\Delta _{1}], [\mathbb{R}] \) выполняются следующие
    равенства: \label{lem:angleexam}
    \begin{enumerate}
        \item \( \varphi \big([\Delta _{1}]\big) = \frac{\pi }{2}
            \),\label{lem:angleexam:1}
        \item \( \varphi_e \big([\Delta _{1}]\big) = \frac{\pi }{3}
            \),\label{lem:angleexam:2}
        \item \( \varphi \big([\mathbb{R}]\big) = \varphi_e
            \big([\mathbb{R}]\big) = \pi  \).\label{lem:angleexam:3}
    \end{enumerate}
\end{lemma}
\begin{proof}
    \ref{lem:angleexam:1}. Из ультраметрического неравенства следует, что \(
    \varphi \big([\Delta _1]\big)\le \frac{\pi }{2} \). Для доказательства
    обратного неравенства найдем угол между двухточечным симплексом \( \Delta
    _{2} = \{x_1, x_2\} \) и трехточечным симплексом \(
    \lambda \Delta _3  = \{y_1,y_2, y_3\}\). Заметим, что если \( R
    \) --- соответствие между \( \Delta _2, \lambda \Delta _3 \),
    то в нем найдутся пары точек \( (x_i,y_j), (x_i,y_k), j \neq k
    \). Значит \( \dis R \ge \big||x_ix_i|-|y_jy_k|\big|  = \lambda
    \). Ясно, что существует \( R \) такое, что \( \dis R = \lambda
    \). Отсюда \( d _{GH}(\Delta _2, \lambda \Delta _3) =
    \frac{\lambda }{2} \). Далее, учитывая, что \( |\Delta _1
    \Delta _2|=\frac{1}{2}, |\Delta _1 \lambda \Delta
    _3|=\frac{\lambda }{2} \), найдем угол между \( \Delta _2,
    \lambda \Delta _3 \):
    \[ \cos \big(\varphi (\Delta _2 , \lambda \Delta _3)\big)
        =\frac{\frac{\lambda ^2}{4} + \frac{1}{4} - \frac{\lambda
        ^2}{4}}{\frac{\lambda }{2}} = \frac{1}{2 \lambda }
        \rightarrow 0, \lambda \rightarrow \infty.
    \]
    Отсюда, \( \varphi \big([\Delta _1]\big) \ge \sup \varphi (\Delta _2,
    \lambda \Delta _3) = \frac \pi 2. \)

    \ref{lem:angleexam:2}. Из ультраметрического неравенства следует, что \(
    \varphi_e \big([\Delta _1]\big)\le \frac{\pi }{3} \). Для доказательства
    обратного неравенства достаточно найти пространства одинакового диаметра,
    расстояние между которыми равно их диаметрам. Этими пространствами
    являются, например, \( \Delta _2, \Delta _3 \).

    \ref{lem:angleexam:3}. Достаточно привести пример пространств \( X,Y\in
    [\mathbb{R}] \), таких, что \( |X \mathbb{R}| = |Y \mathbb{R}| = \frac 1 2
    |XY|. \) Такими пространствами являются \( \mathbb{Z}, \mathbb{R}\times
    [0,1] \).
    %Вставить ссылку на Ивана с названием.

\end{proof}
