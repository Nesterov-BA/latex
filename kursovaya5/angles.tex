В данной секции мы рассмотрим два определения угла облака. Для того,
чтобы дать эти определения сначала необходимо определить угол между
пространствами.
\\
Далее, будем считать, что у облаков нетривиальная стационарная группа
и будем обозначать $ [M] $ - облако с центром $ m $.
\begin{defin}
  Пусть у облака $ [M] $ нетривиальная стационарная группа и его
  центром является $ M $. \textbf{Углом} между пространствами $ X_1,
  X_2 $, $ |X_1 M| = r_1, | X_2, M | =r_2, |X_1 X_2| = d$, где $ r_1,
  r_2 \neq 0 $ называется величина $ \arccos \left(\frac{r_1^2 +
  r_{2}^2 - d^2}{2r_1r_2} \right)$.\\ Будем обозначать его $
  \varphi(X_1, X_2) $.
\end{defin}
\begin{remark}
  Такое определение естественным образом вытекает из теоремы
  косинусов, а именно $ c^{2} = a^2 + b^2 - 2ab\cos (a,b) $.
\end{remark}
У угла между пространствами есть следующее свойство.
\begin{lemma}
  \label{lemmaAngleBetweenSpaces}
  Для любых \( X_1, X_2 \in [M], \lambda \in \St([M]) \) выполняется
  \( \varphi (X_1, X_2) = \varphi (\lambda X_1, \lambda X_2) \).
\end{lemma}
\begin{proof}
  Как в определении угла, будем обозначать \[ |X_1 M| = r_1, | X_2, M
  | =r_2, |X_1 X_2| = d.\]
  По определению угла получаем следующую цепочку равенств:
  \[ \varphi (\lambda X_1, \lambda X_2) = \frac{\lambda ^{2} r_1^2 +
    \lambda ^2 r_2^2 - \lambda^2 d^2}{2 \lambda  r_1 \lambda r_2} =
  \varphi (X_1, X_2).
\]
\end{proof}
Рассмотрим теперь два интересующих нас определения угла облака.
\begin{defin}
  \textbf{Углом} облака $ [M] $ называется величина
  \[
    \varphi \big([M]\big)= \sup \big\{\varphi (X_1, X_2) \bigm| | X_1,M
    |, |X_2,M| \neq 0\big\}.
  \]
\end{defin}
\begin{defin}
  \textbf{Равнобедренным углом} облака $ [M] $ называется величина
  \(
    \varphi_e \big([M]\big) = \sup \big\{\varphi (X_1, X_2) \bigm|
    |X_1,M | = |X_2,M| \neq 0\big\}.
  \)
\end{defin}
Для этих определений получаем следствие из леммы \ref{lemmaAngleBetweenSpaces}.
\begin{lemma}
  Углы облаков обладают следующими свойствами:
  \begin{enumerate}
    \item \(\varphi \big(\lambda [M]\big) = \varphi \big([M]\big)\),
    \item \(\varphi_e \big(\lambda [M]\big) = \varphi_e \big([M]\big)\),
    \item \( \varphi_e \big([M]\big) \le \varphi \big([M]\big) \),
    \item \(0 \le \varphi ([M]) \le \pi\),
    \item \(0 \le \varphi_e ([M]) \le \pi\).
  \end{enumerate}
\end{lemma}
\begin{proof}
  \begin{enumerate}
    \item Следствие леммы \ref{lemmaAngleBetweenSpaces}.
    \item Следствие леммы \ref{lemmaAngleBetweenSpaces}.
    \item В определении равнобедренного угла супремум берется по
      подмножеству пространств из определения угла, следовательно
      равнобедренный угол не больше.
    \item Следует из определения арккосинуса и неравенства треугольника.
    \item Аналогично.
  \end{enumerate}
\end{proof}
Приведем известные примеры углов облаков.
\begin{lemma}
  Для облаков \( [\Delta _{1}], [\mathbb{R}] \) известно следующее:
  \begin{enumerate}
    \item \( \varphi \big([\Delta _{1}]\big) = \frac{\pi }{2} \),
    \item \( \varphi_e \big([\Delta _{1}]\big) = \frac{\pi }{3} \),
    \item \( \varphi \big([\mathbb{R}]\big) = \varphi_e
      \big([\mathbb{R}]\big) = \pi  \).
  \end{enumerate}
\end{lemma}
\begin{proof}
\begin{enumerate}
    \item Из ультраметрического неравенства следует, что \( \varphi \big([\Delta _1]\big)\le \frac{\pi }{2} \). Для доказательства обратного неравенства найдем угол между двухточечным симплексом \( \Delta _{2} = \{x_1, x_2\} \) и трехточечным симплексом \( \lambda \Delta _3  = \{y_1,y_2, y_3\}\). Заметим, что если \( R \) --- соответствие между \( \Delta _2, \lambda \Delta _3 \), то в нем найдутся пары точек \( (x_i,y_j), (x_i,y_k), j \neq k \). Значит \( \dis R \ge \big||x_ix_i|-|y_jy_k|\big|  = \lambda \). Ясно, что существует \( R \) такое, что \( \dis R = \lambda  \). Отсюда \( d _{GH}(\Delta _2, \lambda \Delta _3) = \frac{\lambda }{2} \). Далее, учитывая, что \( |\Delta _1 \Delta _2|=\frac{1}{2}, |\Delta _1 \lambda \Delta _3|=\frac{\lambda }{2} \), найдем угол между \( \Delta _2, \lambda \delta _3 \):
        \[
            \cos \big(\varphi (\Delta _2 , \lambda \Delta _3)\big) =\frac{\frac{\lambda ^2}{4} + \frac{1}{4} - \frac{\lambda ^2}{4}}{\frac{\lambda }{2}} = \frac{1}{2 \lambda } \rightarrow 0, \lambda \rightarrow \infty. 
        \]
        Отсюда, \( \varphi \big([\Delta _1]\big) \ge \sup \varphi (\Delta _2, \lambda \Delta _3) = \frac \pi 2. \)
    \item Из ультраметрического неравенства следует, что \( \varphi_e \big([\Delta _1]\big)\le \frac{\pi }{3} \). Для доказательства обратного неравенства достаточно найти пространства одинакового диаметра, расстояние между которыми равно их диаметрам. Этими пространствами являются, например, \( \Delta _2, \Delta _3 \).
    \item Достаточно привести пример пространств \( X,Y\in [\mathbb{R}] \), таких, что \( |X \mathbb{R}| = |Y \mathbb{R}| = \frac 1 2 |XY|. \) Такими пространствами являются \( \mathbb{Z}, \mathbb{R}\times [0,1] \).
\end{enumerate}
    
\end{proof}
