Далее считаем, что все облака имеют нетривиальную стационарную группу. \\
В евклидовом пространстве косинус угла между двумя векторами определяется по теореме косинусов:
$\|a-b\|^2 = \|a\|^2 + \|b\|^2 - 2\cos(a,b)\|a\|\|b\|$. Если длины векторов равны, то формула превращается в $\|a-b\|^2=2\|a\|^2 - 2\cos(a,b)\|a\|^2$. Таким образом косинус между векторами будет равен $\frac{\|a-b\|^2}{\|a\|^2}-1$. Этой формулой мотивировано следующее определение:

\begin{defin}
  Пусть у облака $[M]$ центр $M$. Также, пусть метрические пространства $X_1,X_2$ находятся на расстоянии $|X_1,M|=|X_2,M|=r$ и $|X_1,X_2|=d$. \emph{Углом} между метрическими пространствами $X_1,X_2$ называется величина $$\varphi(X_1,X_2) = \arccos\left(1-\frac{d^2}{2r^2}\right).$$
  По неравенству треугольника $0\le d^2 \le 4r^2$,откуда $-1 \le 1-\displaystyle\frac{d^2}{2r^2}\le 1$. Это означает, что угол определен корректно.\\
  \emph{Углом раствора} облака $[M]$ называется величина
  \[
    \varphi([M])=\sup\big\{\varphi(X_1,X_2)\colon |X_1,M|=|X_2,M|\big\}.
  \]
\end{defin}
Приведем известные примеры углов раствора:
\begin{enumerate}
    \item $\varphi\big([\Delta_1]\big)=\frac{\pi}{3}$.
    \item $\varphi\big([\mathbb{R}]\big) = \pi$.
\end{enumerate}
Понятие угла раствора обобщает идею неравенства треугольника и ультраметрического неравенства. Следующая лемма является тривиальным свойством угла.
\begin{lemma}
  Для любого облака $[M]$ и $\lambda \in \mathbb{R}^+$ выполнено
  $\varphi\big(\lambda[M]\big) = \varphi\big([M]\big).$
\end{lemma}
\begin{proof}
  Пусть в облаке $[M]$ есть пространства $X_1,X_2$ и угол между ними равен $\varphi(X_1,X_2)$. Это означает, что $1-\frac{|X_1,X_2|^2}{2|X_1,M|^2}=\cos\big(\varphi(X_1,X_2)\big)$. В облаке $\lambda[M]$ соответственно лежат пространства $\lambda X_1, \lambda X_2$, найдем косинус угла между ними:
  \[
    \cos\big(\varphi(\lambda X_1, \lambda X_2)\big) = 1-\frac{|\lambda X_1,\lambda X_2|^2}{2|\lambda X_1,M|^2} = 1- \frac{\lambda^2|X_1,X_2|^2}{2\lambda^2|X_1,M|^2}=\cos \big(\varphi(X_1,X_2)\big).
  \]
  Отображение $\pi_\lambda \colon X \rightarrow \lambda X$ является биекцией между облаками $[M],\lambda [M]$ и сохраняет углы между пространствами. Значит, углы раствора этих облаков равны.
\end{proof}

\begin{theorem}
  Пусть облака $[X],[Y]$ имеют нетривиальное пересечение стационарных групп и $\varphi\big([X]\big)\neq \varphi\big([Y]\big)$. Тогда $\big|[X],[Y]\big|=\infty$.
\end{theorem}

\begin{proof}
  Обозначим углы облаков $ \varphi_X, \varphi_Y $ и $ k_X = \sqrt{2-2\cos \varphi_X},  k_Y = \sqrt{2-2\cos \varphi_Y} $. По определению угла раствора для всякого $ \varepsilon > 0 $ найдутся пространства $ X_1,X_2 $ для которых выполнено $ |X,X_1|=|X,X_2| = r $ и  $ |X_1,X_2|\le (k_X + \varepsilon) r $. Аналогично для облака $[Y]$. Без ограничения общности будем считать, что $ k_Y > k_X $. Найдутся пространства $Y_1, Y_2, X_1, X_2$ для которых выполнено $|X_1,X_2| = k_1 r_X$, $|Y_1,Y_2| = k_2 r_Y$, где $ |X,X_1|=|X,X_2| = r_X $, $ |Y,Y_1|=|Y,Y_2| = r_Y $ и $k_2 \ge k_Y > k_1 \ge k_X $. 

\end{proof}

  
\end{document}
