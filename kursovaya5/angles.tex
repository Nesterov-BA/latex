В данной секции мы рассмотрим два определения угла облака. Для того, чтобы дать эти определения сначала необходимо определить угол между пространствами. 
\\
Далее, будем считать, что у облаков нетривиальная стационарная группа и будем обозначать $ [M] $ - облако с центром $ m $.
\begin{defin}
    Пусть у облака $ [M] $ нетривиальная стационарная группа и его центром является $ M $. \textbf{Углом} между пространствами $ X_1, X_2 $, $ |X_1 M| = r_1, | X_2, M | =r_2, |X_1 X_2| = d$, где $ r_1, r_2 \neq 0 $ называется величина $ \arccos \left(\frac{r_1^2 + r_{2}^2 - d^2}{2r_1r_2} \right)$.\\ Будем обозначать его $ \varphi(X_1, X_2) $.
\end{defin}
\begin{remark}
    Такое определение естественным образом вытекает из теоремы косинусов, а именно $ c^{2} = a^2 + b^2 - 2ab\cos (a,b) $.
\end{remark}
У угла между пространствами есть следующее свойство.
\begin{lemma}
    Для любых \( X_1, X_2 \in [M], \lambda \in \St([M]) \) выполняется \( \varphi (X_1, X_2) = \varphi (\lambda X_1, X_2) \).
\end{lemma} 
\begin{proof}
    Как в определении угла, будем обозначать \[ |X_1 M| = r_1, | X_2, M | =r_2, |X_1 X_2| = d.\] 
\end{proof}
Рассмотрим теперь два интересующих нас определения угла облака.
\begin{defin}
\textbf{Углом} облака $ [M] $ называется величина
\[
    \varphi \big([M]\big)= \sup \big\{\varphi (X_1, X_2) \mid | X_1,M |, |X_2,M| \neq 0\big\}.
\]
\end{defin}
\begin{defin}
    \textbf{Равнобедренным углом} облака $ [M] $ называется величина \[ \varphi_e \big([M]\big) = \sup \big\{\varphi (X_1, X_2) \mid | X_1,M | = |X_2,M| \neq 0\big\}. \]
\end{defin}
Приведем известные примеры углов облаков.
\begin{lemma}
    Для облаков \( [\Delta _{1}], [\mathbb{R}] \) известно следующее:
    \begin{itemize}
    \item \( \varphi \big([\Delta _{1}]\big) = \frac{\pi }{2} \),
    \item \( \varphi_e \big([\Delta _{1}]\big) = \frac{\pi }{3} \),
    \item \( \varphi \big([\mathbb{R}]\big) = \varphi_e \big([\mathbb{R}]\big) = \pi  \).
    \end{itemize}
\end{lemma}
