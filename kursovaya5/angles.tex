Далее считаем, что все облака имеют нетривиальную стационарную группу. \\
В евклидовом пространстве косинус угла между двумя векторами определяется по теореме косинусов:
$\|a-b\|^2 = \|a\|^2 + \|b\|^2 - 2\cos(a,b)\|a\|\|b\|$. Если длины векторов равны, то формула превращается в $\|a-b\|^2=2\|a\|^2 - 2\cos(a,b)\|a\|^2$. Таким образом косинус между векторами будет равен $\frac{\|a-b\|^2}{\|a\|^2}-1$. Этой формулой мотивировано следующее определение:

\begin{defin}
  Пусть у облака $[M]$ центр $M$. Также, пусть метрические пространства $X_1,X_2$ находятся на расстоянии $|X_1,M|=|X_2,M|=r$ и $|X_1,X_2|=d$. \emph{Углом} между метрическими пространствами $X_1,X_2$ называется величина $$\varphi(X_1,X_2) = \arccos\left(1-\frac{d^2}{2r^2}\right).$$
  По неравенству треугольника $0\le d^2 \le 4r^2$,откуда $-1 \le 1-\displaystyle\frac{d^2}{2r^2}\le 1$. Это означает, что угол определен корректно.\\
  \emph{Углом раствора} облака $[M]$ называется величина
  \[
    \varphi([M])=\sup\big\{\varphi(X_1,X_2)\colon |X_1,M|=|X_2,M|\big\}.
  \]
\end{defin}
Приведем известные примеры углов раствора:
\begin{enumerate}
    \item $\varphi\big([\Delta_1]\big)=\frac{\pi}{3}$.
    \item $\varphi\big([\mathbb{R}]\big) = \pi$.
\end{enumerate}
Понятие угла раствора обобщает идею неравенства треугольника и ультраметрического неравенства. Следующая лемма является тривиальным свойством угла.
\begin{lemma}
  Для любого облака $[M]$ и $\lambda \in \mathbb{R}^+$ выполнено
  $\varphi\big(\lambda[M]\big) = \varphi\big([M]\big).$
\end{lemma}
\begin{proof}
  Пусть в облаке $[M]$ есть пространства $X_1,X_2$ и угол между ними равен $\varphi(X_1,X_2)$. Это означает, что $1-\frac{|X_1,X_2|^2}{2|X_1,M|^2}=\cos\big(\varphi(X_1,X_2)\big)$. В облаке $\lambda[M]$ соответственно лежат пространства $\lambda X_1, \lambda X_2$, найдем косинус угла между ними:
  \[
    \cos\big(\varphi(\lambda X_1, \lambda X_2)\big) = 1-\frac{|\lambda X_1,\lambda X_2|^2}{2|\lambda X_1,M|^2} = 1- \frac{\lambda^2|X_1,X_2|^2}{2\lambda^2|X_1,M|^2}=\cos \big(\varphi(X_1,X_2)\big).
  \]
  Отображение $\pi_\lambda \colon X \rightarrow \lambda X$ является биекцией между облаками $[M],\lambda [M]$ и сохраняет углы между пространствами. Значит, углы раствора этих облаков равны.
\end{proof}
\begin{theorem}[Теорема об образе центра ]
  Пусть облака $[X],[Y]$ имеют нетривиальное пересечение стационарных групп и $\varphi\big([X]\big)\neq \varphi\big([Y]\big)$. Также пусть $ R $ - соответствие между этими облаками с конечным искажением равным $ \varepsilon $. Тогда $ \big|R(X),Y\big| \le 2\varepsilon $.
  
\end{theorem}
\begin{proof}
	Нетривиальность стационарной группы $[Y]$ означает, что найдется
	число $l > 1$ такое, что $\{l^{j}|j\in \mathbb{Z}\}$ является подгруппой в
	$\St{[Y]}$.
	
	Зафиксируем $Z$ из образа $X$.
	Предположим, что $|Y,Z| = d > \epsilon$.  Обозначим
	$|Z, kZ| = \rho$, $k \ge 2$, $k = l^{j_{1}}$. По неравенству треугольника $\rho + d \ge kd$,
	откуда $\rho \ge (k-1)d > (k-1)\epsilon$. Тогда $kZ$ лежит в образе
	$M \ne X$. При этом,
	$\rho - \epsilon \le |X, M| \le \rho + \epsilon$. 
	
	Возьмем произвольные $\alpha > 0$ и $\beta \in (0,1)$. Для пространств $(1+\alpha)M, (1-\beta)M$ будут выполняться неравенства:
	$$|X, (1+\alpha)X| = \alpha |X, \Delta_1| \le \alpha\rho + \alpha\epsilon,$$
	$$|X, (1-\beta)X| = \beta|X, \Delta_1| \le \beta\rho + \beta\epsilon,$$
	$$|(1+\alpha) X, (1-\beta)X| = (\alpha + \beta)|X, \Delta_1| \ge (\alpha+\beta)\rho - (\alpha+\beta)\epsilon.$$
	Существуют
	$Y_\alpha, Y_\beta \in [M]$ такие, что
	$kY_\alpha \in R\big((1+\alpha)X\big)$, $kY_\beta \in R\big((1-\beta)X\big)$, и
	для них выполняются следующие неравенства:
	$$|kY, kY_\alpha| \le |X, (1+\alpha)X| + \epsilon \le \alpha\rho + (\alpha+1)\epsilon,$$
	$$|kY, kY_\beta| \le |X, (1-\beta)X| + \epsilon \le \beta\rho + (\beta+1)\epsilon,$$
	$$|kY_\alpha, kY_\beta| \ge |(1+\alpha)X, (1-\beta)X| - \epsilon \ge  (\alpha+\beta)\rho - (\alpha+\beta+1)\epsilon.$$
	Поделим эти неравенства на $k$:
	$$|Y, Y_{\alpha}| \le \frac{\alpha}{k}\rho + \frac{\alpha+1}{k}\epsilon,$$
	$$|Y, Y_{\beta}| \le \frac{\beta}{k}\rho + \frac{\beta+1}{k}\epsilon,$$
	$$|Y_\alpha, Y_{\beta}| \ge \frac{\alpha+\beta}{k}\rho - \frac{\alpha+\beta+1}{k}\epsilon.$$
	и возьмем прообразы пространств $Y, Y_{\alpha}, Y_{\beta}$:
	$$|\Delta_1, X_{\alpha}| \le \frac{\alpha}{k}\rho + \big(\frac{\alpha+1}{k} + 1\big)\epsilon,$$
	$$|\Delta, X_{\beta}| \le \frac{\beta}{k}\rho + \big(\frac{\beta+1}{k}+1\big)\epsilon,$$
	$$|X_\alpha, X_{\beta}| \ge \frac{\alpha+\beta}{k}\rho - \big(\frac{\alpha+\beta+1}{k}+1\big)\epsilon.$$
	Считая, что $\alpha > \beta$ получаем неравенство:
	$$\frac{\alpha+\beta}{k}\rho - \big(\frac{\alpha+\beta+1}{k}+1\big)\epsilon \le \frac{\alpha}{k}\rho + \big(\frac{\alpha+1}{k} + 1\big)\epsilon,$$
	$$\Updownarrow$$
	$$\rho \le \frac{k}{\beta}\bigg(\frac{2\alpha+\beta+2}{k}+2\bigg)\epsilon,$$
	$$\Updownarrow$$
	$$\rho \le \bigg(1+\frac{2\alpha + 2}{\beta} + 2\frac{k}{\beta}\bigg)\epsilon.$$
	Нас интересует оценка сверху для $d$:
	$$d \le \frac{\rho}{k-1} \le \bigg(\frac{1}{k-1}+\frac{2\alpha + 2}{\beta(k-1)} + 2\frac{k}{\beta(k-1)}\bigg)\epsilon. $$
	Последнее слагаемое в скобках строго больше 2 при любых $k>2$, $\alpha>0$,
	$\beta\in (0,1)$, а остальные слагаемые с ростом $k$ стремятся к $0$. Так как стационарная группа нетривиальна, в ней есть последовательности чисел стремящихся к 0 и к $\infty$.
	Устремив $\beta$ к 1, а $k$ к бесконечности получаем оценку:
	$$|Y, M| \le 2\epsilon,$$
	которая завершает доказательство.
	
\end{proof}

\begin{theorem}
  Пусть облака $[X],[Y]$ имеют нетривиальное пересечение стационарных групп и $\varphi\big([X]\big)\neq \varphi\big([Y]\big)$. Тогда $\big|[X],[Y]\big|=\infty$.
\end{theorem}

\begin{proof}
  Предположим, что $ \big|[X],[Y]\big| \neq \infty $. Поскольку облака имеют нетривиальное пересечение стационарных групп, это расстояние может быть равно только $0$. Это означает, что между облаками $ [X], [Y] $ существуют соответствия со сколь угодно малым искажением.

  Обозначим углы облаков $ \varphi_X, \varphi_Y $ и $ k_X = \sqrt{2-2\cos \varphi_X},  k_Y = \sqrt{2-2\cos \varphi_Y} $. По определению угла раствора для всякого $ \varepsilon > 0 $ найдутся пространства $ X_1,X_2 $ для которых выполнено $ |X,X_1|=|X,X_2| = r $ и  $ |X_1,X_2|\le (k_X + \varepsilon) r $. Аналогичное неравенство выполнено для облака $[Y]$. Без ограничения общности будем считать, что $ k_Y < k_X $. Найдутся пространства $Y_1, Y_2, X_1, X_2$ для которых выполнено $|X_1,X_2| = k_1 r_X$, $|Y_1,Y_2| = k_2 r_Y$, где $ |X,X_1|=|X,X_2| = r_X $, $ |Y,Y_1|=|Y,Y_2| = r_Y $ и $k_1 \ge k_X > k_2 \ge k_Y $. 

  Пусть $ R $ - соответствие между облаками $ [X], [Y] $ и $ \dis R = \delta$. Будем обозначать за $ R(A) $ какое-либо пространство из образа $A$ при соответствии. Рассмотрим пространства $ R(X), R(X_1), R(X_2) $. По теореме об образе центра $ \big|Y,R(X)\big| \le 2\delta$
\end{proof}

  
\end{document}
