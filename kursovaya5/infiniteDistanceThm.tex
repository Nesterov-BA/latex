В предыдущей работе была доказана следующая теорема.
\begin{theorem}
  Пусть у облака $[Z]$ нетривиальная стационарная группа, и $Z$
  является его центром. Также, пусть в этом облаке есть пространства $Y_{1},
  Y_{2}$ такие, что $\max\big\{ |Y_{1},Z|, |Y_{2}, Z| \big\} = r>0$, а
  $|Y_{1}, Y_{2}|>r$. Тогда, расстояние между облаками $[\Delta_1]$ и $[Z]$
  равно бесконечности. \label{thrmDist}
\end{theorem}
Следующая теорема является обобщение теоремы \ref{thrmDist}.
\begin{theorem}
  Пусть облака \( [M], [N] \) имеют нетривиальное пересечение
  стационарных групп и их углы \( \varphi \big([M]\big), \varphi
  \big([N]\big) \) различны. Тогда \( d_{GH} \big([M], [N]\big) = \infty \).
\end{theorem}
\begin{proof}
  Без ограничения общности будем считать, что \( \varphi
  \big([M]\big)> \varphi \big([N]\big) \). Для доказательства теоремы
  покажем, что не существует соответствия \( R\in
  \mathcal{R}\big([M],[N]\big) \) с конечным искажением.

  Рассмотрим соответствие \( R \) с конечным искажением \( \dis R = \varepsilon
  < \infty \) как отображение из облака \( [M] \) в облако \( [N] \).
  За \( R(X) \) будем обозначать каком-либо одно пространство из
  образа \( X \) при соответствии.

  По определению угла, в облаке \( [M] \) найдутся пространства \(
  X_1, X_2 \) для которых выполнено:
  \[
    \varphi \big([N]\big) < \varphi (X_1, X_2) \le \varphi \big([M]\big).
  \]
  Обозначим \( |X_1 M| = r_1^X,  |X_2 M| = r_2^X, | X_1 X_2 | =
  d^X\). Также зададим функцию
  \[
    d^N \colon \mathbb{R}^+ \times
    \mathbb{R}^+ \rightarrow \mathbb{R}^+, d^N(r_1, r_2) =
    \sqrt{r_1^2 + r_2^2 - 2r_1r_2\cos\Big(\varphi \big([N]\big)\Big)}.
  \]
  Тогда выполняется следующее неравенство:
  \[
    d^N(r_1^X, r_2^X) < d^X.
  \]
  По определению угла для любых пространств \( Y_1, Y_2 \in [N] \) выполняется
  \begin{equation}
    | Y_1 Y_2 | \le d^N\big(| Y_1N |, | Y_2N |\big).
    \label{ineqAngle}
  \end{equation}
  Итак, наша задача - показать, что найдутся пространства в \( [N] \),
  для которых неравенство выше не выполняется.

  Нетривиальность пересечения стационарных групп означает, что
  найдется подгруппа \[
    \{q^k, k\in \mathbb{Z}, q> 1\} \in \St\big([M]\big) \cap \St \big([N]\big).
  \]
  Обозначим \( |N R(M)| = l \). Будем рассматривать пространства \(
  R(q^k X_1), R(q^k X_2), k \in \mathbb{N} \). Для них выполнены
  следующие неравенства:
  \[
    \big| R(q^k X_1), R(q^k X_2) \big| \ge q^k d^X - \varepsilon,
  \]
  \[
    q^k r_1^X - \varepsilon  \le\big | R(M), R(q^k X_1) \big| \le q^k
    r_1^X + \varepsilon ,
  \]
  \[
    q^k r_2^X - \varepsilon  \le\big | R(M), R(q^k X_2) \big| \le q^k
    r_2^X + \varepsilon .
  \]
  Из последних двух неравенств по неравенству треугольника получаем следующее:
  \[
    q^k r_1^X - l - \varepsilon \le \big|N, R(q^k X_1) \big| \le q^k
    r_1^X + l + \varepsilon,
  \]
  \[
    q^k r_2^X - l - \varepsilon \le \big|N, R(q^k X_2) \big| \le q^k
    r_2^X + l + \varepsilon.
  \]
  Если поделить на \( q^k \) получаем неравенства:
  \[
    \big| q^{-k}R(q^k X_1), q^{-k}R(q^k X_2) \big| \ge d^X -
    \frac{\varepsilon }{q^k},
  \]
  \[
    r_1^X - \frac{l + \varepsilon}{q^k} \le \big|N, q^{-k}R(q^k X_1)
    \big| \le
    r_1^X + \frac{l + \varepsilon}{q^k},
  \]
  \[
    r_2^X - \frac{l + \varepsilon}{q^k} \le \big|N, q^{-k}R(q^k X_2)
    \big| \le
    r_2^X + \frac{l + \varepsilon}{q^k}.
  \]
  Функция \( d^N \) непрерывная, значит
  \[
    \lim_{k \rightarrow \infty
    }d^N\Big(\big|N, q^{-k}R(q^k X_1)
      \big|, \big|N, q^{-k}R(q^k X_2)
    \big|\Big) = d^N(r_1^X, r_2^X).
  \]
  Значит, существует такое \( k_0 \in \mathbb{N} \), что
  \[
    \big| q^{-k_0}R(q^{k_0} X_1), q^{-k_0}R(q^{k_0} X_2) \big| >
    d^N\Big(\big|N, q^{-k_0}R(q^k_0 X_1)
      \big|, \big|N, q^{-k_0}R(q^{k_0} X_2)
    \big|\Big)
  \]
  Пространства \( q^{-k_0}R(q^{k_0}X_1), q^{-k_0}R(q^{k_0}X_2)\) ---
  искомые, для которых не выполняется неравенство \ref{ineqAngle}. Противоречие.
\end{proof}
