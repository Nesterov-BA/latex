Пусть $X$ и $Y$ ---
метрические пространства. Тогда между ними можно задать расстояние, называемое
расстоянием Громова--Хаусдорфа. Введем два его эквивалентных
определения \cite{Lectures}.
\begin{defin}
  Пусть $X$, $Y$ --- метрические пространства. \emph{Соответствием}
  $R$ между этими пространствами
  называется сюръективное многозначное отображение между ними.
  Множество всех соответствий между $X$ и $Y$ обозначается
  $\mathcal{R}(X,Y)$. Также будем отождествлять соответствие и его график.
\end{defin}
\begin{defin}
  Пусть $R$ --- соответствие между $X$ и $Y$.
  \emph{Искажением} соответствия $R$ является величина
  $$ \dis{R} = \sup{\Bigl\{ \big| |xx'| - |yy'| \big| : (x, y), (x',
  y') \in R\Bigr\}}.$$
  Тогда \emph{расстояние Громова--Хаусдорфа} $d_{GH}(X,Y)$ можно определить
  следующим образом
  $$ d_{GH}(X,Y) = \frac{1}{2}\inf \bigl\{\dis{R} : R \in
  \mathcal{R}(X,Y)\bigr\}.$$
  \label{defSootvet}
\end{defin}

Далее, расстояние Громова--Хаусдорфа между метрическими
пространствами $X$ и $Y$ будет обозначаться $|XY|$ или \( | X,Y | \).

Рассмотрим собственный класс всех метрических пространств и
отождествим в нем между собой все метрические пространства,
находящиеся на нулевом расстоянии друг от друга. Обозначим
получившийся класс $\mathcal{GH}_0$.  На нем расстояние
Громова--Хаусдорфа будет являться обобщенной метрикой. Здесь
''обобщенная'', означает, что расстояние между метрическими
пространствами может быть равно бесконечности.
\begin{defin}[\cite{TuzhBog1}] В классе $\mathcal{GH}_{0}$ рассмотрим следующее
  отношение: $X \thicksim Y \Leftrightarrow d_{GH}(X, Y) < \infty$. Нетрудно
  убедиться, что оно будет отношением эквивалентности. Классы этой
  эквивалентности
  называются \emph{облаками}. Облако, в котором лежит метрическое
  пространство $X$
  будем обозначать $[X]$.
\end{defin}

Для любого метрического пространства $X$ определена операция умножения его
на положительное вещественное число $\lambda\colon X\mapsto \lambda X$, а именно
расстояние между любыми точками
пространства умножается на $\lambda$.
\begin{remark} Пусть метрические
  пространства $X$, $Y$ лежат в одном облаке. Тогда
  $d_{GH}(\lambda X, \lambda Y) = \lambda d_{GH}(X,Y) < \infty$, т.е.
  пространства
  $\lambda X$, $\lambda Y$ также будут лежать в одном облаке.
  \label{remOneCloud}
\end{remark}
\begin{defin}Определим операцию умножения облака $[X]$ на
  положительное вещественное
  число $\lambda$ как отображение, переводящее все пространства $Y \in [X]$ в
  пространства $\lambda Y$. По замечанию $\ref{remOneCloud}$ все
  полученные пространства будут
  лежать в облаке $[\lambda X]$.
\end{defin}
При таком отображении облако может как
измениться, как было показано в \cite{TuzhBog2}, так и перейти в
себя. Для последнего случая вводится специальное
определение.
\begin{defin}[\cite{TuzhBog2}]
  Стационарной группой $\St\bigl([X]\bigr)$ облака $[X]$
  называется подмножество $\mathbb{R}_+$, то есть множества всех
  положительных вещественных чисел, такое, что для всех
  $\lambda \in \St\bigl([X]\bigr)$, $[X] = [\lambda X]$. Полученное подмножество
  действительно будет подгруппой в $\mathbb{R}_+$. Тривиальной
  будем называть стационарную группу, равную $\{1\}$. Пересечение
  двух стационарных групп называем нетривиальным, если оно не равно \( \{1\} \).
\end{defin}

\begin{lemma}[\cite{TuzhBog2}]
  В каждом облаке с нетривиальной стационарной группой существует единственное
  пространство $X$ такое, что для любого $\lambda$ из стационарной группы
  выполняется $X = \lambda X$.
  \label{centerLemma}
\end{lemma}
\begin{defin}
  Пространство из леммы \ref{centerLemma} будем называть \emph{центром} облака.
\end{defin}
\begin{remark} В
  облаке $[\Delta_1]$ для любого пространства $X$ выполняется:
  $$|\lambda X \mu X| = |\lambda - \mu||X\Delta_1|.$$
\end{remark}
\begin{remark}[Ультраметрическое неравенство] В облаке
  $[\Delta_{1}]$ для всех пространств $X_{1}, X_{2}$ выполняется неравенство:
  $$|X_{1}X_{2}| \le \max\big\{|X_{1} \Delta_{1}|,|X_{2}\Delta_{1}|\big\}.$$
  \label{remUltraMetric}
\end{remark}
