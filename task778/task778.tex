\documentclass[a4paper,12pt]{article}

\usepackage{cmap}
\usepackage{amsthm}
\usepackage{amsmath}
\usepackage{amssymb}					% поиск в PDF
\usepackage[T2A]{fontenc}			% кодировка
\usepackage[utf8]{inputenc}			% кодировка исходного текста
\usepackage[english,russian]{babel}	% локализация и переносы
\usepackage[margin=0.25in]{geometry}
\usepackage{pgfplots}
\pgfplotsset{width=10cm,compat=1.9}

% We will externalize the figures
\date{\today}
\begin{document}
\section{Задача 7.78}
Пространства $ (Y,\rho_Y),(Z,\rho_Z) $. \\Пространства функций 
$ \big(C(Y),L_{\rho_Y}\big), $  $\big(C(Z),L_{\rho_Z}\big) $. Рассмотрим $ f\in C(Y) $, $ f(y_i)=a_i $.\\
Тогда $ L_{\rho_Y}(f) = \max\left\{\frac{|a_i,a_j|}{\rho_Y(y_i,y_j)},i\neq j\right\} = \max\left\{|a_2-a_1|,|a_3-a_2|\right\}$.\\
Аналогично для $ g\in C(Z), g(z_i) = b_i $ выполнено $ L_{\rho_Z}(g) = \frac{|b_2-b_1|}{3} .$
Теперь рассмотрим соответствующие пространства состояний\\ $ \mathcal{S}\big(C(Y)\big) ,\mathcal{S}\big(C(Z)\big)$. Так как квантовые пространства представляют из себя функции на дискретных пространствах, $ \mathcal{S}\big(C(Y)\big) = \{c_1 \delta_{y_1} + c_2 \delta_{y_2} + c_1 \delta_{y_3} \colon c_1+c_2+c_3=1, c_i\ge 0\}$,
$ \mathcal{S}\big(C(Z)\big) = \{t \delta_{z_1} + (1-t) \delta_{z_2}\colon t \in [0,1]\}$. Найдем расстояния между элементами этих пространств:
\begin{multline*}
  \rho_{L_{\rho_Z}}\big(t_1\delta_{z_1} + (1-t_1)\delta_{z_2},t_2\delta_{z_1} + (1-t_2)\delta_{z_2}\big) =\\
  = \sup\big\{|t_1b_1 + (1-t_1)b_2 - t_2b_1 - (1-t_2)b_2| \colon |b_2-b_1|\le 3\big\}=\\
  = \sup\big\{|(t_2-t_1)(b_2-b_1)\colon |b_2-b_1|\le 3\big\}=3|t_2-t_1|
\end{multline*}

Пространство состояний $ \mathcal{S}\big(C(Z)\big) $ является отрезком длиной $ 3 $. Аналогично показывается, что в пространстве $ \mathcal{S}\big(C(Y)\big) $ точки $ \delta_{y_1},\delta_{y_2},\delta_{y_3} $ соединены отрезками $ \big|\left[\delta_1,\delta_2\right]\big|=\big|\left[\delta_2,\delta_3\right]\big|=1,\big|\left[\delta_1,\delta_3\right]\big|=2 .$

Далее, пусть два элемента $ \mathcal{S}\big(C(Y)\big)$ равны $ c_1 \delta_{y_1} + c_3 \delta_{y_3}  + (1-c_1 - c_3)\delta_{y_2} $, $ \widetilde{c}_1 \delta_{y_1} + \widetilde{c}_3 \delta_{y_3}  + (1-\widetilde{c}_1 - \widetilde{c}_3)\delta_{y_3} $. Мы параметризуем каждый элемент двумя числами из множества $\{(c_1,c_3) \colon c_1 + c_3 \le 1, c_1\ge 0, c_3 \ge 0\}$. Расстояние между ними равно 
\begin{multline*}
  \sup\big\{|c_1 a_1 + c_3a_3+(1-c_1-c_3)a_2 - \widetilde{c}_1a_1-\widetilde{c}_3a_3 - (1-\widetilde{c}_1-\widetilde{c}_3)a_2|\\ \colon |a_2-a_1|\le1, |a_3-a_2|\le 1\big\} = \\
 = \sup\big\{|(\widetilde{c}_1 - c_1)(a_2-a_1) + (\widetilde{c}_3-c_3)(a_2-a_3)| \colon|a_2-a_1|\le1, |a_3-a_2|\le 1 \big\} =\\
  = |\widetilde{c}_1-c_1 |+| \widetilde{c}_3 - c_3|.
\end{multline*}  

Пространство состояний $ \mathcal{S}\big(C(Y)\big) $ изоморфно треугольнику с вершинами $(0,0),(0,1),(1,0)$ в $ \mathbb{R}^2 $ с метрикой $ l^1 $. 
Для пространств 






\end{document}
