%%%%%%%%%%%%%%%%%
% This is an sample CV template created using altacv.cls
% (v1.3, 10 May 2020) written by LianTze Lim (liantze@gmail.com). Now
% compiles with pdfLaTeX, XeLaTeX and LuaLaTeX.
% This fork/modified version has been made by Nicolás Omar González
% Passerino (nicolas.passerino@gmail.com, 15 Oct 2020)
%
%% It may be distributed and/or modified under the
%% conditions of the LaTeX Project Public License, either version 1.3
%% of this license or (at your option) any later version.
%% The latest version of this license is in
%%    http://www.latex-project.org/lppl.txt
%% and version 1.3 or later is part of all distributions of LaTeX
%% version 2003/12/01 or later.
%%%%%%%%%%%%%%%%

%% If you need to pass whatever options to xcolor
\PassOptionsToPackage{dvipsnames}{xcolor}

%% If you are using \orcid or academicons
%% icons, make sure you have the academicons
%% option here, and compile with XeLaTeX
%% or LuaLaTeX.
% \documentclass[10pt,a4paper,academicons]{altacv}

%% Use the "normalphoto" option if you want a normal photo instead of
% cropped to a circle
% \documentclass[10pt,a4paper,normalphoto]{altacv}

%% Fork: CV dark mode toggle enabler to use a inverted color palette.
%% Use the "darkmode" option if you want a color palette used to
% \documentclass[10pt,a4paper,darkmode]{altacv}

\documentclass[10pt,a4paper,ragged2e,withhyper]{altacv}

%% AltaCV uses the fontawesome5 and academicons fonts
%% and packages.
%% See http://texdoc.net/pkg/fontawesome5 and
% http://texdoc.net/pkg/academicons for full list of symbols. You
% MUST compile with XeLaTeX or LuaLaTeX if you want to use academicons.

% Change the page layout if you need to
\geometry{left=1.2cm,right=1.2cm,top=1cm,bottom=1cm,columnsep=0.75cm}

% The paracol package lets you typeset columns of text in parallel
\usepackage{paracol}
\usepackage[english, russian]{babel}
\usepackage{setspace}
% Change the font if you want to, depending on whether
% you're using pdflatex or xelatex/lualatex
\ifxetexorluatex
% If using xelatex or lualatex:
\setmainfont{Roboto Slab}
\setsansfont{Lato}
\renewcommand{\familydefault}{\sfdefault}
\else
% If using pdflatex:
\usepackage[rm]{roboto}
\usepackage[defaultsans]{lato}
% \usepackage{sourcesanspro}
\renewcommand{\familydefault}{\sfdefault}
\fi

% Fork: Change the color codes to test your personal variant on any mode
\ifdarkmode%
\definecolor{PrimaryColor}{HTML}{0F52D9}
\definecolor{SecondaryColor}{HTML}{3F7FFF}
\definecolor{ThirdColor}{HTML}{F3890B}
\definecolor{BodyColor}{HTML}{ABABAB}
\definecolor{EmphasisColor}{HTML}{ABA2A2}
\definecolor{BackgroundColor}{HTML}{242424}
\else%
\definecolor{PrimaryColor}{HTML}{475C08}
\definecolor{SecondaryColor}{HTML}{6A8911}
\definecolor{ThirdColor}{HTML}{E0F120}
\definecolor{BodyColor}{HTML}{666666}
\definecolor{EmphasisColor}{HTML}{2E2E2E}
\definecolor{BackgroundColor}{HTML}{F9F9F9}
\fi%

\colorlet{name}{PrimaryColor}
\colorlet{tagline}{PrimaryColor}
\colorlet{heading}{PrimaryColor}
\colorlet{headingrule}{ThirdColor}
\colorlet{subheading}{SecondaryColor}
\colorlet{accent}{SecondaryColor}
\colorlet{emphasis}{EmphasisColor}
\colorlet{body}{BodyColor}
\pagecolor{BackgroundColor}

% Change some fonts, if necessary
\renewcommand{\namefont}{\Huge\rmfamily\bfseries}
\renewcommand{\personalinfofont}{\small\bfseries}
\renewcommand{\cvsectionfont}{\LARGE\rmfamily\bfseries}
\renewcommand{\cvsubsectionfont}{\large\bfseries}

% Change the bullets for itemize and rating marker
% for \cvskill if you want to
\renewcommand{\itemmarker}{{\small\textbullet}}
\renewcommand{\ratingmarker}{\faCircle}

%% sample.bib contains your publications
%% \addbibresource{sample.bib}

\begin{document}
\name{Нестеров Борис}
\tagline{Студент математик}
%% You can add multiple photos on the left or right
%% \photoL{2cm}{FullSizeRender 3.jpg}
%% \photoR{2cm}{IMG_1122.JPG}
\photoL {2cm}{IMG_6720.jpg}

\personalinfo{
  \printinfo{\faBirthdayCake}{26.04.2002}\\
  \email{nesterov.boris123@gmail.com}
  %\homepage{nicolasomar.me}
  %\medium{nicolasomar}
  %% You MUST add the academicons option to \documentclass, then
  % compile with LuaLaTeX or XeLaTeX, if you want to use \orcid or
  % other academicons commands.
  % \orcid{0000-0000-0000-0000}
  %% You can add your own arbtrary detail with
  %% \printinfo{symbol}{detail}[optional hyperlink prefix]
  \printinfo{\faPhone}{+79998261105}
  %% Or you can declare your own field with
  %% \NewInfoFiled{fieldname}{symbol}[optional hyperlink prefix] and use it:
  % \NewInfoField{gitlab}{\faGitlab}[https://gitlab.com/]
  \github{Nesterov-BA}
}

\makecvheader
%% Depending on your tastes, you may want to make fonts of itemize
% environments slightly smaller
% \AtBeginEnvironment{itemize}{\small}

%% Set the left/right column width ratio to 6:4.
\columnratio{0.25}

% Start a 2-column paracol. Both the left and right columns will automatically
% break across pages if things get too long.
\begin{paracol}{2}
  % ----- STRENGTHS -----
  \cvsection{Навыки}

  \cvtag{Теория Вероятностей}
  \cvtag{Статистика}
  \cvtag{Мат. Анализ}
  \cvtag{A/B тестирование}
  \\
  \cvtag{Линейная Алгебра}
  \\
  \cvtag{Численные Методы}
  \cvtag{Случайные процессы}

  % ----- STRENGTHS -----

  % ----- LANGUAGES -----
  \cvsection{Языки}
  \cvlang{Английский}{C1}\\
  \divider

  %% Yeah I didn't spend too much time making all the
  %% spacing consistent... sorry. Use \smallskip, \medskip,
  %% \bigskip, \vpsace etc to make ajustments.
  \cvlang{Русский}{Носитель}\\
  \smallskip
  % ----- LANGUAGES -----

  % ----- LEARNING -----
  \cvsection{
    \begin{spacing}{1.0}Технические навыки
  \end{spacing}}
  \cvtag{Python}
  \cvtag{NumPy}
  \cvtag{Pandas}
  \cvtag{Sklearn}
  \cvtag{Matplotlib}
  \cvtag{SQL}
  \cvtag{C++}
  \cvtag{Excel}
  % ----- LEARNING -----

  % ----- OTHER -----
  \cvsection{}
  \cvtag{Основы ML}
  \cvtag{Photoshop}
  \cvtag{Linux}
  \cvtag{LaTex}
  % ----- OTHER -----

  % ----- HOBBIES -----

  % ----- HOBBIES -----

  % ----- MOST PROUD -----
  % \cvsection{Most Proud of}

  % \cvachievement{\faTrophy}{Fantastic Achievement}{and some details
  % about it}\\
  % \divider
  % \cvachievement{\faHeartbeat}{Another achievement}{more details
  % about it of course}\\
  % \divider
  % \cvachievement{\faHeartbeat}{Another achievement}{more details
  % about it of course}
  % ----- MOST PROUD -----

  % \cvsection{A Day of My Life}

  % Adapted from @Jake's answer from http://tex.stackexchange.com/a/82729/226
  % \wheelchart{outer radius}{inner radius}{
  % comma-separated list of value/text width/color/detail}
  % \wheelchart{1.5cm}{0.5cm}{%
  %   6/8em/accent!30/{Sleep,\\beautiful sleep},
  %   3/8em/accent!40/Hopeful novelist by night,
  %   8/8em/accent!60/Daytime job,
  %   2/10em/accent/Sports and relaxation,
  %   5/6em/accent!20/Spending time with family
  % }

  % use ONLY \newpage if you want to force a page break for
  % ONLY the current column
  \newpage

  %% Switch to the right column. This will now automatically move to the second
  %% page if the content is too long.
  \switchcolumn
  \cvsection{Проекты}
  \cvevent{Численные Методы}{| Механико-Математический факультет}{}{}
  \begin{itemize}
    \item Численное решение системы дифференциальных уравнений
    \item Численное решение уравнений с частными производными
    \item Численное приближение функций многих переменных
  \end{itemize}
  \divider

  \cvevent{Научная Работа}{| Механико-Математический
  факультет}{Сентябрь 2023 -- Н.В.}{}
  \begin{itemize}
    \item Написание и публикация статьи по метрической геометрии в
      коллаборации с научным руководителем

  \end{itemize}

  % ----- EXPERIENCE -----
  \cvsection{Опыт работы}
  \cvevent{Репетитор}{}{Сентябрь 2022 -- Н.В.}{}
  \begin{itemize}
    \item Подготовка школьников к математическим олимпиадам,
      экзаменам и поступлению в университеты
    \item Занятия с учениками младших курсов технических факультетов
  \end{itemize}
  \divider

  \cvevent{Преподаватель математики }{| Школа №54}{Сентябрь 2021 --
  Май 2022}{Москва, Россия}
  \begin{itemize}
    \item Проведение еженедельного кружка по подготовке к олимпиадам
      по математике
    \item Составление листков с задачами
  \end{itemize}

  \vspace{3pt}
  % ----- EXPERIENCE -----

  % ----- EDUCATION -----
  \cvsection{Образование}
  \cvevent{Специалитет}{| МГУ имени М. В. Ломоносова}{Сентябрь 2019
  -- Н.В.}{Москва, Россия}
  Специалист по направлению "Фундаментальная математика и механика"

  \divider

  \cvevent{Доп. образование}{| Институт ИИ МГУ}{Сентябрь 2024 --
  Н.В.}{Москва, Россия}
  AI Masters, программа ''Data Science / Data Engineering''

  % ----- EDUCATION -----
  % ----- SPHERES -----

\end{paracol}
\end{document}
