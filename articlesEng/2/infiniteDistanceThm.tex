В работе \cite{nesterov2025gromovhausdorffdistancecloudbounded} была
доказана следующая теорема.
\begin{thm}
  Пусть в облаке \( [Z] \) есть пространства $Y_{1}$, $ Y_{2}$ такие,
  что \\$\max\big\{ |Y_{1}Z|, |Y_{2} Z| \big\} = r>0$, а $|Y_{1}
  Y_{2}|>r$. Тогда, расстояние между облаками $[\Delta_1]$ и $[Z]$
  равно бесконечности. \label{thrmDist}
\end{thm}
Следующая теорема является обобщением теоремы \ref{thrmDist}.
\begin{thm}
  Пусть облака \( [M], [N] \) имеют нетривиальное пересечение
  стационарных групп и их углы \( \varphi \big([M]\big), \varphi
  \big([N]\big) \) различны. Тогда \( d_{GH} \big([M], [N]\big) =
  \infty \).
\end{thm}
\begin{proof}
  Без ограничения общности будем считать, что \( \varphi \big([M]\big)>
  \varphi \big([N]\big) \). Для доказательства теоремы покажем, что не
  существует соответствия \( R\in \mathcal{R}\big([M],[N]\big) \) с
  конечным искажением.

  Предположим противное, т. е. существует соответствие \( R \) с
  конечным искажением \( \dis R = \varepsilon < \infty \) как
  отображение из облака \( [M] \) в облако \( [N] \). Построим функцию
  \( f \colon [M] \rightarrow [N] \), \( f(X) \) -- произвольное
  пространство из \(
    R(X)
  \). По определению угла, в облаке \( [M] \) найдутся пространства \(
  X_1, X_2 \), для которых выполнено
  \[
    \varphi \big([N]\big) < \varphi (X_1, X_2) \le \varphi
  \big([M]\big). \]
  Положим \( |X_1 M| = r_1^X,  |X_2 M| = r_2^X, | X_1 X_2 | = d^X\).
  Также зададим функцию
  \[ d^N \colon \mathbb{R}^+ \times \mathbb{R}^+ \rightarrow
    \mathbb{R}^+, d^N(r_1, r_2) =
    \sqrt{r_1^2 + r_2^2 - 2r_1r_2\cos\Big(\varphi \big([N]\big)\Big)}.
  \]
  Тогда выполняется следующее неравенство:
  \begin{equation}
    d^N(r_1^X, r_2^X) < d^X.
    \label{ineqdx}
  \end{equation}
  По определению угла для любых пространств \( Y_1, Y_2 \in [N] \)
  выполняется
  \begin{equation}
    | Y_1 Y_2 | \le d^N\big(| Y_1N |, | Y_2N |\big). \label{ineqAngle}
  \end{equation}
  Итак, наша задача --- показать, что найдутся пространства в \( [N]
  \), для которых неравенство (\ref{ineqAngle}) не выполняется.

  Нетривиальность пересечения стационарных групп означает, что найдется
  подгруппа \[
    \{q^k \colon k\in \mathbb{Z}, q> 1\} \in \St\big([M]\big) \cap \St
    \big([N]\big).
  \]
  Положим \( |N R(M)| = l \). Будем рассматривать пространства \(
  R(q^k X_1), R(q^k X_2), k \in \mathbb{N} \). Для них выполнены
  следующие неравенства:
  \begin{equation}
    \big| R(q^k X_1), R(q^k X_2) \big| \ge q^k d^X - \varepsilon,
    \label{ineqDistx1x2}
  \end{equation}
  \[
    q^k r_1^X - \varepsilon  \le\big | R(M), R(q^k X_1) \big| \le q^k
    r_1^X + \varepsilon ,
  \]
  \[
    q^k r_2^X - \varepsilon  \le\big | R(M), R(q^k X_2) \big| \le q^k
    r_2^X + \varepsilon .
  \]
  Из последних двух неравенств по неравенству треугольника получаем
  следующее:
  \begin{equation}
    q^k r_1^X - l - \varepsilon \le \big|N, R(q^k X_1) \big| \le
    q^k r_1^X + l + \varepsilon,\label{ineqDistx1N}
  \end{equation}
  \begin{equation}
    q^k r_2^X - l - \varepsilon \le \big|N, R(q^k X_2) \big| \le q^k
    r_2^X + l + \varepsilon.\label{ineqDistx2N}
  \end{equation}
  Поделим неравенства (\ref{ineqDistx1x2}), (\ref{ineqDistx1N}),
  (\ref{ineqDistx2N}) на \( q^k \):
  \[ \big| q^{-k}R(q^k X_1), q^{-k}R(q^k X_2) \big| \ge d^X -
  \frac{\varepsilon }{q^k}, \]
  \[
    r_1^X - \frac{l + \varepsilon}{q^k} \le \big|N, q^{-k}R(q^k X_1)
    \big| \le
    r_1^X + \frac{l + \varepsilon}{q^k},
  \]
  \[
    r_2^X - \frac{l + \varepsilon}{q^k} \le \big|N, q^{-k}R(q^k X_2)
    \big| \le
    r_2^X + \frac{l + \varepsilon}{q^k}.
  \]
  Функция \( d^N \) непрерывная, значит
  \[
    \lim_{k \rightarrow \infty
    }d^N\Big(\big|N, q^{-k}R(q^k X_1)
      \big|, \big|N, q^{-k}R(q^k X_2)
    \big|\Big) = d^N(r_1^X, r_2^X).
  \]
  Из неравенства (\ref{ineqdx}) следует, что найдутся \( \delta
    _1, \delta _2
  > 0 \) такие, что \( d^N(r_1^X,r_2^X)<d^N(r_1^X, r_2^X)+\delta
  _1 < d^X- \delta _2 < d^X \). Выберем такое \( k_1 \), чтобы
  для всех \( k > k_1 \) выполнялось
  \[
    d^N\Big(\big|N, q^{-k}R(q^k X_1)
      \big|, \big|N, q^{-k}R(q^k X_2)
    \big|\Big) \in \big(d^N(r_1^X, r_2^X)- \delta _1,
  d^N(r_1^X, r_2^X)+ \delta _1\big).  \]
  Также выберем \( k_2 \), такое что \( \big| q^{-k}R(q^k X_1),
  q^{-k}R(q^k X_2) \big| > d^X - \delta _2 \).
  Теперь, если взять \( k_0 = \max\{k_1,k_2\} \), то
  \[ \big| q^{-k_0}R(q^{k_0} X_1), q^{-k_0}R(q^{k_0} X_2) \big|
    > d^N\Big(\big|N, q^{-k_0}R(q^k_0 X_1) \big|, \big|N,
  q^{-k_0}R(q^{k_0} X_2) \big|\Big) \]
  Пространства \( q^{-k_0}R(q^{k_0}X_1), q^{-k_0}R(q^{k_0}X_2)\)
  ---
  искомые, для которых не выполняется неравенство (\ref{ineqAngle}).
  Противоречие.
\end{proof}
