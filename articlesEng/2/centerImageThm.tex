В предыдущей работе была доказана следующая теорема.
\begin{thm}
	Пусть $M$ -- центр облака $[M]$, имеющего нетривиальную
	стационарную группу. $R$ -- соответствие между $[\Delta_{1}]$ и $[M]$ с конечным
	искажением $\varepsilon$. Тогда образ пространства $\Delta_{1}$ лежит от $M$ на
	расстоянии не большем $2 \varepsilon $.
	\label{thrmCenterImage}
\end{thm}
В ее доказательстве используется следующий факт об облаке \( [\Delta _1] \).
\begin{thm}
    Для всякого ограниченного пространства \( X \) луч \( \lambda X \), \(
    \lambda \in [0,\infty ) \) является геодезической. Иначе говоря, для всяких
    \( \lambda _1, \lambda _2 \in [0,\infty) \) выполняется \(| \lambda _1 X,
    \lambda _2 X| = | \lambda _1 - \lambda _2 | \cdot | X, \Delta _1 |\).
\end{thm}
В общем случае это неверно, в частности, в облаке \( [\mathbb{R}]
\) луч \( \lambda \mathbb{Z} \) не является геодезической
(\cite{mikhailov2025newgeodesiclinesgromovhausdorff}).
