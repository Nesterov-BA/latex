\documentclass[leqno]{article}
\usepackage{geometry}
\usepackage{graphicx}
\usepackage[cp1251]{inputenc}
%\usepackage[russian]{babel}

\usepackage[english]{babel}
\usepackage{mathtools}
\usepackage{amsfonts,amssymb,mathrsfs,amscd,amsmath,amsthm}

\usepackage{verbatim}

\usepackage{url}

%\usepackage[dvipdfmx]{hyperref}

\usepackage{stmaryrd}
\usepackage{chebsb}
\ifpdf
\usepackage{hyperref}
\else
\usepackage[dvipdfmx]{hyperref}
\fi

\begin{document}
\title{On the Gromov--Hausdorff distance between the cloud of bounded
metric spaces and a cloud with nontrivial stabilizer}
\author{B.\,A.~Nesterov}
\date{}
\maketitle

\begin{abstract}
  The paper studies the class of all metric spaces considered up to
  zero Gromov--Hausdorff distance between them. In this class, we
  examine clouds --- classes of spaces situated at finite
  Gromov--Hausdorff distances from a reference space. We prove that
  all clouds are proper classes. The Gromov--Hausdorff distance is
  defined for clouds similarly with the case of that for metric
  spaces. A multiplicative group of transformations of clouds is
  defined which is called stabilizer.%% Explain stabilizer
  We show that under certain restrictions the distance between the
  cloud of bounded metric spaces and a cloud with a nontrivial
  stabilizer is finite. In particular, the distance between the cloud
  of bounded metric spaces and the cloud containing the real line is calculated.

  {\bf Keywords:} metric spaces, Gromov--Hausdorff distance, clouds,
  proper class
\end{abstract}

\section{Intodiction}
\markright{\thesection.~Introduction}

The present paper is devoted to the study of the Gromov--Hausdorff
distance ~\cite{Edwards, Gromov81, Gromov99},
defined on the proper class of all non-empty metric spaces considered
up to isometry. It is known that in this class, the distance is a
generalized pseudometric i.e., the distance may be zero for non-equal
elements, and the distance can take infinite values.

Traditionally, the Gromov--Hausdorff distance is studied on the class
of compact metric spaces considered up to isometry. This class is
called the Gromov--Hausdorff space. On it, the distance becomes a
metric. Below, the Gromov--Hausdorff distance between spaces $ X $
and $ Y $ will be denoted by $ d_{GH}(X, Y) $ or $ |X, Y| $.

M. Gromov introduced this distance in \cite{Gromov81} and used it to
prove the theorem on groups of polynomial growth.

Later, this distance found applications in computational geometry,
where it was used for shape matching and similarity measurement
\cite{memoli1}. The Gromov--Hausdorff distance can also be applied in
robotics for motion planning \cite{robotics}.

Computing the Gromov--Hausdorff distance is algorithmically an
NP-hard problem, and to simplify calculations, the distance is often
modified, see, for example, \cite{memoli2}.

In \cite{Gromov99}, M. Gromov also considered the Gromov--Hausdorff
distance on classes of unbounded spaces that are at a finite distance
from each other. In the present paper these classes will be called
\emph{clouds} and are the primary subject of the present work. Gromov
stated that all clouds are complete and contractible but he did not
provide proofs for these facts \cite{Gromov99}. Subsequently, S. A.
Bogatyi and A. A. Tuzhilin in \cite{TuzhBog1} proved the completeness
of clouds. However, the problem of contractibility turned out to be
significantly more challenging.

First, we note that contractibility is a topological concept, as it
relies on continuous mappings. Recall that in the von
Neumann-Bernays-G\"odel (NBG) set theory, every object is a class,
which can either be a set or a proper class. By definition, a set is
an element of another class \cite{Neumann, Bernays, Godel}. A proper
class cannot ba an element of another class. Thus, for proper
classes, it is impossible to define a topology in the usual sense,
since the class itself would then have to be an element of it. In
\cite{BorIvTuzh1}, a generalized notion of topology and continuous
mappings for proper classes was introduced using the concept of
filtration by sets. If such a filtration exists in a class, the class
is called \emph{topological}. In the present work, we prove that
every cloud is a proper class. Therefore, to meaningfully discuss the
contractibility of clouds, a generalization of topology is required,
as, for example, developed in \cite{BorIvTuzh1}.

However, generalizing topology alone is not sufficient. To illustrate
this, we introduce several additional concepts.
For any metric space, one can define an operation of multiplication
by a positive real number $\lambda$. Under this operation
$H_{\lambda}\colon X \to \lambda X$, all distances in the metric
space $X$ are scaled by $\lambda$. Moreover, in the case of bounded
metric spaces, this operation can be extended to zero by setting $0
\cdot X := \Delta_1$, where $\Delta_1$ is a one-point metric space.

It is well known that for any bounded spaces $X$, $Y$ and
non-negative real numbers $\lambda$, $\mu$, the following holds:
$$
|\lambda X, \mu X| = |\lambda - \mu| \cdot |X, \Delta_1| =
\frac{1}{2} |\lambda - \mu| \cdot \text{diam}\, X,
$$
where $\text{diam}\, X$ is the diameter of $X$. Also, for any metric
spaces $X,Y$ the following holds:
$$
|\lambda X, \lambda Y| = \lambda |X, Y|,
$$

From these properties, it is straightforward to show that the cloud
of bounded metric spaces is indeed contractible.

However, if we consider a cloud containing $\mathbb{R}^n$, the
operation $H_\lambda$ maps the cloud into itself for all $\lambda$
but it is discontinuous at certain points. Moreover, there exist
spaces that, when multiplied by some positive real numbers, are
mapped to spaces at infinite Gromov--Hausdorff distance
\cite{TuzhBog1}. This implies that the clouds containing them are not
preserved under such scaling.
From the properties mentioned above, it follows that if multiplying a
space by $\lambda$ keeps it within its own cloud, then all spaces in
that cloud also remain within it. Moreover, if a space transitions
into another cloud under such scaling, then all spaces from the
source cloud transition to the same target. Thus, the operation of
multiplication by $\lambda$ can be naturally extended to clouds themselves.

As discussed earlier, the mapping $H_\lambda$ posesses nontrivial
properties, which motivates further investigation. To study the
operation $H_\lambda$, the concept of a \emph{stabilizer} of a cloud
was introduced: it is the multiplicative group of all positive
$\lambda$ for which $H_\lambda$ maps the cloud into itself. In
\cite{TuzhBog2}, the notion of a \emph{center} of a cloud was defined
as a space that, under the action of transformations from the
stabilizer, maps to a space at zero Gromov--Hausdorff distance from
itself. It was also shown that every cloud has a unique center up to
zero distances. The concepts of the stabilizer and the center of a
cloud play a key role in this work.

The present study primarily focuses on investigating the
Gromov--Hausdorff distance between clouds, one of which is the cloud
of bounded metric spaces. We formulate and prove a theorem concerning
the image of $\Delta_1$ under a correspondence with finite distortion
between the cloud of bounded metric spaces and a cloud with a
nontrivial stabilizer. As a corollary, we establish a theorem stating
that the distance from the cloud of bounded metric spaces to clouds
with nontrivial stabilizers with some additional assumptions is
infinite. As an example, we show that the Gromov--Hausdorff distance
between the cloud of bounded metric spaces and the cloud which
contains the real line is infinite.

The author expresses deep gratitude to their advisor, A. A. Tuzhilin,
and Professor A. O. Ivanov for formulating the problem and for
fruitful discussions of the results.

\section{Preliminaries}
\markright{\thesection.~Preliminaries}

Let $ X $ and $ Y $ be metric spaces. A distance between them, called
the \emph{Gromov--Hausdorff distance}, can be defined. Below, we
present two equivalent definitions \cite{Lectures}.

\begin{defin}
  Let $ X $ and $ Y $ be metric spaces. A \emph{correspondence} $ R $
  between these spaces is a surjective multivalued mapping from $ X $
  to $ Y $. The set of all correspondences between $ X $ and $ Y $ is
  denoted by $ \mathcal{R}(X,Y) $. We will also identify a
  correspondence with its graph.
\end{defin}

\begin{defin}
  Let $ R $ be a correspondence between $ X $ and $ Y $. The
  \emph{distortion} of $ R $ is defined as
  $$
  \dis{R} = \sup \Bigl\{ \big| |xx'| - |yy'| \big| : (x, y), (x', y')
  \in R \Bigr\}.
  $$
  Then, the \emph{Gromov--Hausdorff distance} $ d_{GH}(X,Y) $ can be defined as
  $$
  d_{GH}(X,Y) = \frac{1}{2} \inf \bigl\{ \dis{R} : R \in
  \mathcal{R}(X,Y) \bigr\}.
  $$
  \label{defSootvet}
\end{defin}
\begin{defin}
  Let $X,Y$ be subsets of a metric space $Z$. We define the Hausdorff
  distance between $X,Y$ as follows:
  $$
  d_H(X, Y) \coloneqq \max \left\{ \sup_{x \in X} d(x, Y), \sup_{y
  \in Y} d(X, y) \right\}.
  $$
\end{defin}
\begin{defin}
  A \emph{realization} of a pair of metric spaces $ (X,Y) $ is a
  triple $ (X', Y', Z) $ of metric spaces such that:
  $ X' \subset Z $ and $ Y' \subset Z $,
  $ X' $ is isometric to $ X $ and
  $ Y' $ is isometric to $ Y $.

  The \emph{Gromov--Hausdorff distance} $ d_{GH}(X,Y) $ between $ X $
  and $ Y $ is the infimum of all numbers $ r $ for which there
  exists a realization $ (X', Y', Z) $ satisfying $ d_H(X', Y') \leq
  r $, where $ d_H $ is the Hausdorff distance.
\end{defin}

Henceforth, the Gromov--Hausdorff distance between metric spaces $ X
$ and $ Y $ will be denoted by $ |X,Y| $.

We will consider two metric spaces equivalent if they are a zero
Gromov--Hausdorff distance from each other. The resulting class is
denoted by $ \mathcal{GH}_0 $. On this class, the Gromov--Hausdorff
distance becomes a \emph{generalized metric}.

\begin{defin}[\cite{TuzhBog1}]
  In the class $\mathcal{GH}_{0}$, we consider the following
  relation: $X \thicksim Y \Leftrightarrow d_{GH}(X, Y) < \infty$. It
  is easy to verify that this is an equivalence relation. The
  equivalence classes of this relation are called \emph{clouds}. The
  cloud containing a metric space $X$ will be denoted by $[X]$.
\end{defin}

For any metric space $X$, we can define an operation of
multiplication by a positive real number $\lambda\colon X\mapsto
\lambda X$, specifically $(X, \rho) \mapsto (X, \lambda \rho)$, where
the distance between any two points of the space is scaled by a
factor of $\lambda$.

\begin{remark}
  Let metric spaces $X$, $Y$ belong to the same cloud. Then
  $d_{GH}(\lambda X, \lambda Y) = \lambda d_{GH}(X,Y)\allowbreak <
  \infty$, which means the spaces $\lambda X$, $\lambda Y$ will also
  belong to the same cloud.
  \label{remOneCloud}
\end{remark}

\begin{defin}
  We define the operation of multiplying a cloud $[X]$ by a positive
  real number $\lambda$ as the mapping that transforms all spaces $Y
  \in [X]$ into spaces $\lambda Y$. According to Remark
  \ref{remOneCloud}, all resulting spaces will belong to the cloud
  $[\lambda X]$.
\end{defin}

Under such a mapping, a cloud may either change or remain invariant.
For the latter case, we introduce a special definition.

\begin{defin}[\cite{TuzhBog2}]
  The \emph{stabilizer} $\St\bigl([X]\bigr)$ of a cloud $[X]$ is the
  subset of $\mathbb{R}_+$ such that for all $\lambda \in
  \St\bigl([X]\bigr)$, $[X] = [\lambda X]$. This subset forms a
  subgroup of $\mathbb{R}_+$. The stabilizer is called \emph{trivial}
  when it equals $\{1\}$.
\end{defin}

Let us provide several examples of clouds and their stabilizers.

\begin{itemize}
  \item Let $\Delta_1$ be a one-point metric space. Then
    $\St\bigl([\Delta_1]\bigr) = \mathbb{R}_+$.
  \item $\St\bigl([\mathbb{R}]\bigr) = \mathbb{R}_+$.
  \item Suppose a function $\phi(n)$ satisfies $\lim\limits_{n
    \rightarrow \infty} \phi(n + 1) - \phi(n) = +\infty$. For $q >
    1$, define the space $X_q = \left\{q^{\phi(n)}: n \in
    \mathbb{N}\right\}$. Then $\St\bigl([X_q]\bigr) = \{1\}$ \cite{TuzhBog1}.
  \item For a natural number $p$, define the space $X_p = \left\{p^n
    : n \in \mathbb{Z}\right\}$. For any prime $p$, we have
    $\St\bigl([X_p]\bigr) = \left\{p^n : n \in \mathbb{Z}\right\}$
    \cite{BogBog1}.
\end{itemize}

\begin{lemma}[\cite{TuzhBog2}]
  In every cloud with a nontrivial stabilizer, there exists a unique
  space $X$ such that for any $\lambda$ from the stabilizer, $X =
  \lambda X$ holds.
  \label{centerLemma}
\end{lemma}

\begin{defin}
  The metric space from the Lemma \ref{centerLemma} will be called
  the \emph{center} of the cloud.
\end{defin}
\begin{remark} For any metric space $X$ in the cloud $[\Delta_1]$:
  $$|\lambda X, \mu X| = |\lambda - \mu|\cdot|X,\Delta_1|.$$
\end{remark}
\begin{remark}[Ultrametric inequaliy] For any metric spaces $X_{1},
  X_{2}$ from the cloud$[\Delta_{1}]$ the following inequality holds:
  $$|X_{1},X_{2}| \le \max\{|X_{1}, \Delta_{1}|,|X_{2},\Delta_{1}|\}.$$
  \label{remUltraMetric}
\end{remark}
\section{Cloud cardinality}
\markright{\thesection.~Cloud cardinality}

By definition, metric spaces are sets. Therefore, to extend the
Gromov--Hausdorff distance construction to clouds, we must either
establish that clouds are sets or appropriately modify the distance definition.

We employ the lemma concerning the nature of cardinal number sets:

\begin{lemma}[\cite{levySet}]
  Any set of cardinal numbers has an upper bound.
\end{lemma}

The following corollary from this lemma will be necessary for our proof:

\begin{corollary}\label{colCardinal}
  The class of unbounded cardinals is proper.
\end{corollary}

We now formulate and prove the theorem about the class of spaces
within each cloud:

\begin{theorem}
  All clouds are proper classes.
\end{theorem}

\begin{proof}
  Following Corollary \ref{colCardinal}, it suffices to show that any
  cloud contains spaces of arbitrarily large cardinality. Let $X$ be
  a metric space of cardinality $\alpha$. We extend this space to one
  of greater cardinality $\beta > \alpha$ by constructing $X_\beta =
  X \cup \Delta_\beta$, where $\Delta_\beta$ is a simplex of
  cardinality $\beta$.

  Fix an arbitrary point $x \in X$ and set the distance from $x$ to
  any simplex point as 1. For $x' \in X$ and $y \in \Delta_\beta$, define:
  $$
  \rho_{X_\beta}(y,x') = \rho_{X_\beta}(x',y) := \rho_X(x',x) + 1.
  $$
  All other distances remain unchanged.

  The resulting space $X_\beta$ is indeed metric:
  \begin{enumerate}
    \item Symmetry and non-negativity are obvious
    \item The triangle inequality holds for all cases:
      \begin{itemize}
        \item When $x', z' \in \Delta_\beta$ or $x', z' \in X$: trivial.
        \item When $x' \in X$, $z' \in \Delta_\beta$:
          \begin{itemize}
            \item if $y' \in X$: $\rho_{X_\beta}(x',z') =
              \rho_X(x,x') + 1 \leq \rho_X(x,y') + \rho_X(y',x') + 1$,
            \item if $y' \in \Delta_\beta$: $\rho_{X_\beta}(x',z') =
              \rho_X(x,x') + 1 \leq \rho_X(x',x) + 2$.
          \end{itemize}
      \end{itemize}
  \end{enumerate}

  Since $X$ can be isometrically embedded into $X_\beta$ with
  $X_\beta$ lies in a closed 1-neighborhood of $X$, their
  Gromov--Hausdorff distance is finite.
\end{proof}

\begin{remark}
  As all clouds are proper classes, a bijection exists between any
  two clouds. This implies, in particular, that the class of
  correspondences between any two clouds is non-empty.
\end{remark}

\begin{defin}
  Let $\mathcal{R}([X],[Y])$ denote the class of all correspondences
  between clouds $[X]$ and $[Y]$. We define the distortion
  $\text{dis}\, R$ as in Definition \ref{defSootvet}. The
  Gromov--Hausdorff distance between clouds is:
  $$
  d_{GH}([X],[Y]) = \frac{1}{2}\inf\{\text{dis}\, R : R \in
  \mathcal{R}([X],[Y])\}.
  $$
\end{defin}

\section{Center image theorem}
\markright{\thesection.~Center image theorem}

Before proving the main theorem of this section we must list several
key properties of correspondences between clouds on which our proof will rely.
%% Написать все обозначения и пояснения
\begin{lemma}\label{lemDiamImage}
  If $R$ is a correspondence between two clouds with distortion $r$
  then for any spaces $Y_1,Y_2$ which lie in $R(X)$, $|Y_1,Y_2| \le r$.
\end{lemma}

\begin{proof}
  If spaces $Y_{1}, Y_{2}$ lie in the image of $X$, then
  $$
  \operatorname{dis} R \geq \big|\left|Y_{1},Y_{2}\right| -
  \left|X,X\right|\big| = |Y_{1}, Y_{2}|,
  $$
  from which $\operatorname{diam} R(X) \leq \operatorname{dis} R$ follows.
\end{proof}

\begin{corollary}
  Suppose than $R$ is a correspondence between two clouds with
  distortion $r$ and $Y_1,Y_2$ are metric spaces such that $|Y_1,Y_2|
  > r$. Then $Y_1$ and $Y_2$ cannot lie in the image of a single metric space.
\end{corollary}
Now we are ready to formulate and prove the theorem about the image
of $\Delta_1$.
\begin{theorem}\label{thrmCenterImage}
  Let $M$ be the center of the cloud $[M]$ with a nontrivial
  stabilizer. Let $R$ be a correspondence between $[\Delta_{1}]$ and
  $[M]$ with finite distortion $\varepsilon$. Then any space from the
  image $R(\Delta_{1})$ lies at a distance from $M$ not exceeding
  $2\varepsilon$.
\end{theorem}

\begin{proof}
  The nontriviality of the stabilizer $[M]$ implies that there exists
  a number $l > 1$ such that $\{l^{j} \mid j\in \mathbb{Z}\}$ is a
  subgroup of $\operatorname{St}([M])$.

  Fix $Y$ in the image of $\Delta_{1}$. Suppose that $|M, Y| = d >
  \varepsilon$. Denote $|Y, kY| = \rho$, where $k \geq 2$, $k =
  l^{j_{1}}$. By the triangle inequality $\rho + d \geq kd$, hence
  $\rho \geq (k-1)d > (k-1)\varepsilon$. Then $kY$ lies in the image
  of $X \neq \Delta_1$, with $\rho - \varepsilon \leq |X, \Delta_1|
  \leq \rho + \varepsilon$.

  Take arbitrary $\alpha > 0$ and $\beta \in (0,1)$. For spaces
  $(1+\alpha)X$, $(1-\beta)X$ the following inequalities hold:
  \begin{align*}
    |X, (1+\alpha)X| &= \alpha |X, \Delta_1| \leq \alpha\rho +
    \alpha\varepsilon, \\
    |X, (1-\beta)X| &= \beta|X, \Delta_1| \leq \beta\rho + \beta\varepsilon, \\
    |(1+\alpha) X, (1-\beta)X| &= (\alpha + \beta)|X, \Delta_1| \geq
    (\alpha+\beta)\rho - (\alpha+\beta)\varepsilon.
  \end{align*}

  There exist $Y_\alpha, Y_\beta \in [M]$ such that $kY_\alpha \in
  R\big((1+\alpha)X\big)$, $kY_\beta \in R\big((1-\beta)X\big)$, and
  for them the following inequalities hold:
  \begin{align*}
    |kY, kY_\alpha| &\leq |X, (1+\alpha)X| + \varepsilon \leq
    \alpha\rho + (\alpha+1)\varepsilon, \\
    |kY, kY_\beta| &\leq |X, (1-\beta)X| + \varepsilon \leq \beta\rho
    + (\beta+1)\varepsilon, \\
    |kY_\alpha, kY_\beta| &\geq |(1+\alpha)X, (1-\beta)X| -
    \varepsilon \geq (\alpha+\beta)\rho - (\alpha+\beta+1)\varepsilon.
  \end{align*}

  Dividing these inequalities by $k$, we obtain:
  \begin{align*}
    |Y, Y_{\alpha}| &\leq \frac{\alpha}{k}\rho +
    \frac{\alpha+1}{k}\varepsilon, \\
    |Y, Y_{\beta}| &\leq \frac{\beta}{k}\rho + \frac{\beta+1}{k}\varepsilon, \\
    |Y_\alpha, Y_{\beta}| &\geq \frac{\alpha+\beta}{k}\rho -
    \frac{\alpha+\beta+1}{k}\varepsilon.
  \end{align*}

  Taking preimages of spaces $Y, Y_{\alpha}, Y_{\beta}$, we obtain:
  \begin{align*}
    |\Delta_1, X_{\alpha}| &\leq \frac{\alpha}{k}\rho +
    \left(\frac{\alpha+1}{k} + 1\right)\varepsilon, \\
    |\Delta, X_{\beta}| &\leq \frac{\beta}{k}\rho +
    \left(\frac{\beta+1}{k}+1\right)\varepsilon, \\
    |X_\alpha, X_{\beta}| &\geq \frac{\alpha+\beta}{k}\rho -
    \left(\frac{\alpha+\beta+1}{k}+1\right)\varepsilon.
  \end{align*}

  Assuming $\alpha > \beta$ we obtain:
  $$
  \frac{\alpha+\beta}{k}\rho -
  \left(\frac{\alpha+\beta+1}{k}+1\right)\varepsilon \leq
  \frac{\alpha}{k}\rho + \left(\frac{\alpha+1}{k} + 1\right)\varepsilon,
  $$
  which is equivalent to:
  $$
  \rho \leq \frac{k}{\beta}\left(\frac{2\alpha+\beta+2}{k}+2\right)\varepsilon,
  $$
  and further:
  $$
  \rho \leq \left(1+\frac{2\alpha + 2}{\beta} +
  2\frac{k}{\beta}\right)\varepsilon.
  $$

  We are interested in an upper bound for $d$:
  $$
  d \leq \frac{\rho}{k-1} \leq \left(\frac{1}{k-1}+\frac{2\alpha +
  2}{\beta(k-1)} + 2\frac{k}{\beta(k-1)}\right)\varepsilon.
  $$

  The last term in parentheses is strictly greater than 2 for any
  $k>2$, $\alpha>0$, $\beta\in (0,1)$, while the other terms tend to
  0 as $k$ increases. Since the stabilizer is nontrivial, it contains
  sequences of numbers tending to both 0 and $\infty$. Taking $\beta$
  to 1 and $k$ to infinity, we obtain the estimate:
  $$
  |Y, M| \leq 2\varepsilon,
  $$
  which completes the proof.
\end{proof}

\section{Cloud of the real line and the non-utlrametric unequality}
\markright{\thesection.~Cloud of the real line and the
non-utlrametric unequality}

\begin{lemma}\label{lemmaDiamDist}
  If $X$ is a subset of the real line and $\mathbb{R}\setminus X$
  contains an interval of diameter $2d$, then $|\mathbb{R}, X| \ge d$.
\end{lemma}

\begin{proof}
  Suppose that
  $\mathbb{R} \setminus X$ contains an interval $(a-d, a+d)$. Suppose
  $|\mathbb{R}, X|<d$. Let $(\mathbb{R}^{\prime}, X^{\prime}, Y)$ be
  a realization of $(\mathbb{R}, X)$ with $d_H(\mathbb{R}^{\prime},
  X^{\prime}) = d^{\prime}<d$. Define
  $$
  U_1 := \bigcup_{\substack{x \in X^{\prime} \\ x \leq a-d}}
  B\left(x, d^{\prime}+\frac{d-d^{\prime}}{2}\right), \quad
  U_2 := \bigcup_{\substack{x \in X^{\prime} \\ x \geq a+d}}
  B\left(x, d^{\prime}+\frac{d-d^{\prime}}{2}\right),
  $$
  so $U_1 \cup U_2$ covers $X^{\prime}$ with balls of radius
  $d^\prime + \frac{d-d^{\prime}}{2}$.

  Then $U_1$ and $U_2$ are two disjoint open sets, but
  $\mathbb{R}^{\prime} \subset U_1 \cup U_2$, which contradicts the
  connectedness of the real line.
\end{proof}

For the cloud $[\Delta_{1}]$, Remark \ref{remUltraMetric} shows that
the ultrametric inequality holds. The following lemma demonstrates
that this inequality may fail for the cloud $[\mathbb{R}]$.

Consider $\mathbb{R}$ as a subset of $\mathbb{R}^2$ and add the point
$(0,1)$ with distances given by the $L_1$ metric in $\mathbb{R}^2$.
Denote this space by $\widetilde{\mathbb{R}}$.

\begin{theorem}\label{thrmRUltraMetric}
  For the spaces $\mathbb{Z}$ and $\widetilde{\mathbb{R}}$, the
  following hold\/\rom{:}
  \begin{enumerate}
    \item $|\mathbb{Z}, \mathbb{R}| \le \frac{1}{2}$,
      $|\widetilde{\mathbb{R}}, \mathbb{R}| \le \frac{1}{2}$.\label{thrmPt:1}
    \item $|\mathbb{Z}, \widetilde{\mathbb{R}}|>  \frac{1}{2}$.\label{thrmPt:2}
  \end{enumerate}
\end{theorem}

\begin{proof}
  (1) Embedding $\mathbb{Z}$ in $\mathbb{R}$ gives a realization with
  Hausdorff distance $\frac{1}{2}$. For $\widetilde{\mathbb{R}}$,
  embed it naturally in $\mathbb{R}^2$ and $\mathbb{R}$ as $\{(x,
  \frac{1}{2}) : x \in \mathbb{R}\}$; the Hausdorff distance is again
  $\frac{1}{2}$.

  (2) Let $R$ be a correspondence between $\mathbb{Z}$ and
  $\widetilde{\mathbb{R}}$ with distortion $1 + \varepsilon$, where
  $(0,1)$ is in the image of some $i \in \mathbb{Z}$. By Lemma
  \ref{lemDiamImage}, the image of $i$ must lie in $(-\varepsilon,
  \varepsilon) \times \{0\} \cup \{(0,1)\}$.

  Let $\mathcal{N}$ be the set of integers whose images lie in
  $(-\varepsilon, \varepsilon) \times \{0\} \cup \{(0,1)\}$. Then
  $\mathbb{Z} \setminus \mathcal{N}$ is at \mbox{distance $\geq 1$}
  from $\mathbb{R}$ by Lemma \ref{lemmaDiamDist}.

  Define $R'$ by removing from $R$:
  \begin{itemize}
    \item The pair $(i,(0,1))$,
    \item all pairs $(k,x)$ with $x \in (-\varepsilon, \varepsilon)
      \times \{0\}$.
  \end{itemize}

  Then $R'$ is a correspondence between
  $\mathbb{R}\setminus(-\varepsilon, \varepsilon) \times \{0\}$ and
  $\mathbb{Z} \setminus \mathcal{N}$ with distortion:
  $$
  1+\varepsilon \geq \operatorname{dis} R' \geq
  2\big|\mathbb{R}\setminus(-\varepsilon, \varepsilon), \mathbb{Z}
  \setminus \mathcal{N}\big|.
  $$
  By the triangle inequality:
  $$
  2\big|\mathbb{R}\setminus(-\varepsilon, \varepsilon), \mathbb{Z}
  \setminus \mathcal{N}\big| \geq 2(1 - \varepsilon).
  $$
  This yields $1+\varepsilon \geq 2-2\varepsilon$, hence $\varepsilon
  \geq \frac{1}{3}$ and consequently:
  $$
  |\widetilde{\mathbb{R}}, \mathbb{Z}| \geq \frac{2}{3} > \frac{1}{2}.
  $$
\end{proof}

\section{Gromov--Hausdorff distance betweem clouds}
\markright{\thesection.~Gromov--Hausdorff distance betweem clouds}

We present a lemma about the distance between clouds with
intersecting stabilizers.

\begin{lemma}\label{lemmaDist}
  If two clouds have a nontrivial intersection of their stabilizers,
  then the Gromov--Hausdorff distance between them can only be $0$ or $\infty$.
\end{lemma}

\begin{proof}
  For any clouds $[X], [Y]$ and any $\lambda \in \mathbb{R}^{+}$, we have
  $$
  \big|\lambda[X], \lambda[Y]\big| = \lambda\big|[X], [Y]\big|.
  $$
  If $\lambda \neq 1$ belongs to the stabilizers of both clouds, then
  $$
  \big|[X],[Y]\big| = \big|\lambda[X], \lambda[Y]\big| =
  \lambda\big|[X], [Y]\big|.
  $$
  Since $\lambda \neq 1$, the quantity $\big|[X],[Y]\big|$ can only
  be $0$ or $\infty$.
\end{proof}

\begin{theorem}\label{thrmDist}
  Let $[Z]$ be a cloud with a nontrivial stabilizer which has $Z$ as
  its center. Suppose there exist spaces $Y_{1}, Y_{2} \in [Z]$ such that
  \begin{enumerate}
    \item $\max\{ |Y_{1},Z|, |Y_{2}, Z| \} = r > 0$,
    \item $|Y_{1}, Y_{2}| > r$.
  \end{enumerate}
  Then the Gromov--Hausdorff distance between clouds $[\Delta_1]$ and
  $[Z]$ is infinite.
\end{theorem}

\begin{proof}
  The clouds $[\Delta_1]$ and $[Z]$ have stabilizers with nontrivial
  intersection. By Lemma \ref{lemmaDist}, the distance between them
  can only be $0$ or $\infty$.

  To prove the theorem, it suffices to show that the
  Gromov--Hausdorff distance cannot be $0$. We need to establish that
  no correspondence with arbitrarily small distortion can exist
  between them. Let $R$ be a correspondence between $[\Delta_1]$ and
  $[Z]$ with $\operatorname{dis} R = \varepsilon < \infty$.

  Fix $Y \in R(\Delta_1)$. By Theorem \ref{thrmCenterImage}, the
  Gromov--Hausdorff distance between $Y$ and $Z$ is at most $2\varepsilon$.

  The theorem's conditions give:
  $$
  \max\{ |Y_{1},Z|, |Y_{2}, Z| \} = r < |Y_{1}, Y_{2}|.
  $$
  This implies there exists $c > 0$ such that $|Y_{1},Y_{2}| = (1 +
  c)r$. Consider the preimages $X_1 \in R^{-1}(Y_{1})$, $X_2 \in
  R^{-1}(Y_{2})$. We obtain:
  $$
  |X_1, \Delta_1| \leq |Y_{1}, Y| + \varepsilon \leq r + 3\varepsilon.
  $$
  The same is true for $X_2$, while:
  $$
  |X_1, X_2| \geq |Y_{1}, Y_{2}| - \varepsilon = (1+c)r - \varepsilon.
  $$
  By Remark \ref{remUltraMetric}:
  $$
  (1+c)r - \varepsilon \leq r + 3\varepsilon.
  $$
  which yields:
  $$
  \varepsilon \geq \frac{cr}{4}.
  $$
  This lower bound for $\varepsilon = \operatorname{dis} R$ shows the
  distortion cannot be arbitrarily small, hence the Gromov--Hausdorff
  distance cannot be $0$ and must therefore be infinite.
\end{proof}

\begin{corollary}
  In the cloud $[\mathbb{R}]$, we can take $Y_{1} = \mathbb{Z}$ and
  $Y_{2} = \widetilde{\mathbb{R}}$. By Theorem
  \ref{thrmRUltraMetric}, they satisfy the conditions of Theorem
  \ref{thrmDist} with $r = \frac{1}{2}$. The stabilizer of
  $[\mathbb{R}]$ is $\mathbb{R}^{+}$ (nontrivial). Therefore
  $\big|[\Delta_{1}],[\mathbb{R}]\big| = \infty$.
\end{corollary}
\markright{References}

%библиография по Гарвардскому стандарту
\begin{thebibliography}{99}

  \bibitem{Edwards} D. Edwards, \emph{The Structure of Superspace.
  In: Studies in Topology}, ed. by Stavrakas N.M. and Allen K.R., New
  York, London, San Francisco, Academic Press, 1975.

  \bibitem{Gromov81} M. Gromov, \emph{Structures m\'etriques pour les
  vari\'et\'es riemanniennes}, edited by Lafontaine and Pierre Pansu, 1981.

  \bibitem{Gromov99} M. Gromov, \emph{Metric structures for
  Riemannian and non-Riemannian spaces}, Birkh\"auser, ISBN
  0-8176-3898-9 (translation with additional content), 1999

  \bibitem{memoli1}
  F. M\'emoli and S. Guillermo, \emph{Comparing point clouds}, In
  Proceedings of the 2004 Eurographics/ACM SIGGRAPH symposium on
  Geometry processing (SGP '04). Association for Computing Machinery,
  New York, NY, USA, 32-40. https://doi.org/10.1145/1057432.1057436, 2004.

  \bibitem{robotics} F. Sukkar, J. Wakulicz, K. M. B. Lee, W. Zhi and
  R. Fitch, \emph{Multi-query Robotic Manipulator Task Sequencing
  with Gromov--Hausdorff Approximations}, ArXiv e-prints,
  arXiv:2209.04800, 2024.

  \bibitem{memoli2}
  F. M\'emoli, \emph{Gromov--Hausdorff distances in Euclidean
  spaces}, IEEE Computer Society Conference on Computer Vision and
  Pattern Recognition Workshops, Anchorage, AK, USA, pp. 1-8, 2008.

  \bibitem{TuzhBog1}
  S.\,A. Bogatyy and A.\,A. Tuzhilin, \emph{Gromov--Hausdorff class:
  its completeness and cloud geometry}, ArXiv e-prints,
  arXiv:2110.06101, 2021.

  \bibitem{Neumann}
  J. von Neumann, \emph{Eine Axiomatisierung der Mengenlehre},
  Journal f\"ur die Reine und Angewandte Mathematik, 1925.

  \bibitem{Bernays}
  P. Bernays, \emph{A System of Axiomatic Set Theory--Part I}, The
  Journal of Symbolic Logic, doi:10.2307/2268862, JSTOR 2268862, 1937.

  \bibitem{Godel}
  K. G\"odel, \emph{The Consistency of the Axiom of Choice and of the
    Generalized Continuum Hypothesis with the Axioms of Set Theory
  (Revised ed.)}, Princeton University Press,  ISBN 978-0-691-07927-1, 1940.

  \bibitem{BorIvTuzh1}
  S.\,I. Borzov, A.\,O. Ivanov and A.\,A. Tuzhilin,
  \emph{Extendability of Metric Segments in Gromov--Hausdorff
  Distance}, ArXiv e-prints, arXiv:2009.00458, 2020.

  \bibitem{TuzhBog2}
  S.\,I. Bogataya, S.\,A. Bogatyy, S.\,A., Redkozubov, V.\,V. \&
  Tuzhilin A.\,A. , Clouds in Gromov--Hausdorff Class: their
  completeness and centers, ArXiv e-prints, arXiv:2202.07337, 2022.

  \bibitem{Lectures}
  D. Burago, Y.\,D. Burago, and S.\,O. Ivanov, \emph{A Course in
  Metric Geometry}, American Mathematical Soc., 2001.

  \bibitem{BogBog1}
  S.\,I. Bogataya and S.\,A. Bogatyy, \emph{Isometric Cloud
  Stabilizer}, Topology and its Applications, Volume 329, 2023.

  \bibitem{levySet}A. Levy, \emph{Basic set theory}, Perspectives in
  mathematical logic, Springer-Verlag, Berlin, Heidelberg, and New York, 1979.

\end{thebibliography}

\end{document}
